\begin{consigna}{red}
	Existen muchos programas y recursos \textit{online} para hacer diagramas de Gantt, entre los cuales destacamos:

	\begin{itemize}
		\item Planner
		\item GanttProject
		\item Trello + \textit{plugins}. En el siguiente link hay un tutorial oficial: \\
		      \url{https://blog.trello.com/es/diagrama-de-gantt-de-un-proyecto}
		\item Creately, herramienta online colaborativa.
		      \\\url{https://creately.com/diagram/example/ieb3p3ml/LaTeX}
		\item Se puede hacer en latex con el paquete \textit{pgfgantt}\\
		      \url{http://ctan.dcc.uchile.cl/graphics/pgf/contrib/pgfgantt/pgfgantt.pdf}
	\end{itemize}

	Pegar acá una captura de pantalla del diagrama de Gantt, cuidando que la letra
	sea suficientemente grande como para ser legible. Si el diagrama queda
	demasiado ancho, se puede pegar primero la ``tabla'' del Gantt y luego pegar la
	parte del diagrama de barras del diagrama de Gantt.

	Configurar el software para que en la parte de la tabla muestre los códigos del
	EDT (WBS).\\ Configurar el software para que al lado de cada barra muestre el
	nombre de cada tarea.\\ Revisar que la fecha de finalización coincida con lo
	indicado en el Acta Constitutiva.

	En la figura \ref{fig:gantt}, se muestra un ejemplo de diagrama de gantt
	realizado con el paquete de \textit{pgfgantt}. En la plantilla pueden ver el
	código que lo genera y usarlo de base para construir el propio.

	Las fechas pueden ser calculadas utilizando alguna de las herramientas antes
	citadas. Sin embargo, el siguiente ejemplo fue elaborado utilizando
	\href{https://docs.google.com/spreadsheets/d/1fBz8NhSpc4tkkhz3KjJCbh1nR_ltDkfEcZi4tZXduqs}{esta
		hoja de cálculo}.

	Es importante destacar que el ancho del diagrama estará dado por la longitud
	del texto utilizado para las tareas (Ejemplo: tarea 1, tarea 2, etcétera) y el
	valor \textit{x unit}. Para mejorar la apariencia del diagrama, es necesario
	ajustar este valor y, quizás, acortar los nombres de las tareas.

	\begin{figure}[htpb]
		\begin{center}
			\begin{ganttchart}[
					time slot unit=day,
					time slot format=isodate,
					x unit=0.038cm,
					y unit title=0.7cm,
					y unit chart=0.6cm,
					milestone/.append style={xscale=4}
				]{2021-03-05}{2021-12-16}
				\gantttitlecalendar*{2021-03-05}{2021-12-16}{year} \\
				\gantttitlecalendar*{2021-03-05}{2021-12-16}{month} \\
				\ganttgroup{Duración Total}{2021-03-05}{2021-12-16} \\
				%%%%%%%%%%%%%%%%%Organización
				\ganttgroup{Organización}{2021-03-05}{2021-04-16} \\
				\ganttbar{Planificación del proyecto}{2021-03-05}{2021-04-15} \\
				%%%%%%%%%%%%%%%%%Ejecución
				\ganttgroup{Ejecución}{2021-04-16}{2021-10-21} \\
				\ganttbar{Tarea 1}{2021-04-16}{2021-04-29} \\
				\ganttbar{Tarea 2}{2021-04-30}{2021-05-13} \\
				\ganttbar{Tarea 3}{2021-05-14}{2021-05-27} \\
				\ganttbar{Tarea 4}{2021-05-28}{2021-07-12} \\
				\ganttbar{Tarea 5}{2021-07-13}{2021-08-09} \\
				\ganttbar{Tarea 6}{2021-08-10}{2021-09-23} \\
				\ganttbar{Tarea 7}{2021-09-24}{2021-09-30} \\
				\ganttbar{Tarea 8}{2021-10-01}{2021-10-14} \\
				\ganttbar{Tarea 9}{2021-10-15}{2021-10-21} \\
				% %%%%%%%%%%%%%%%%%Finalización
				\ganttgroup{Finalización}{2021-10-22}{2021-12-16} \\
				\ganttbar{Memoria v1}{2021-10-22}{2021-11-04} \\
				\ganttbar{Memoria v2}{2021-11-05}{2021-11-18} \\
				\ganttbar{Memoria final}{2021-11-19}{2021-12-02} \\
				% La fecha del siguiente milestone es la fecha en que terminamos la memoria
				\ganttmilestone{Enviar memoria al director}{2021-12-02} \\
				\ganttbar{Elaborar la presentación}{2021-12-03}{2021-12-16} \\
				\ganttmilestone{Ensayo de la presentación}{2021-12-16} \\
				%%%%%%%%%%%%%%%%%%%%%%%%%%%%%%%%%%%%%%%%%%%%%%%%%%%%%%%%%%%%%%%
			\end{ganttchart}
		\end{center}
		\caption{Diagrama de gantt de ejemplo}
		\label{fig:gantt}
	\end{figure}

	\begin{landscape}
		\begin{figure}[htpb]
			\centering
			\includegraphics[height=.85\textheight]{./Figuras/Gantt-2.png}
			\caption{Ejemplo de diagrama de Gantt (apaisado).} %Modificar este título acorde.
			\label{fig:diagGantt}
		\end{figure}

	\end{landscape}

\end{consigna}