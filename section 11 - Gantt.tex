\subsection{Resumen}

En la tabla \ref{tab:tabGantt}, se puede visualizar un resumen del detalle de
la cantidad de horas, la fecha de inicio y fin de cada grupo de tareas
estimadas a dedicar en el desarrollo del proyecto.

\begin{table}[ht]
	\begin{tabularx}{\linewidth}{|p{0.8cm}|p{8.33cm}|p{1cm}|p{1.8cm}|p{1.8cm}|}
		\hline
		%\rowcolor[HTML]{C0C0C0}
		%WBS                           & Tarea                                             & Horas           & Inicio     & Fin        \\ \hline
		%\textbf{ID} & \textbf{Tarea} & \textbf{Duración (días)} & \textbf{Fecha Inicio} & \textbf{Fecha Fin} \\ \hline
		\multicolumn{5}{|c|}{Etapa 1}                                                                                                       \\ \hline
		\rowcolor[HTML]{C0C0C0}
		WBS                                 & Tarea                                             & Horas           & Inicio     & Fin        \\ \hline
		\centering{1}                       & Planificación del proyecto                        & \raggedleft{50} & 20-08-2024 & 18-09-2024 \\ \hline
		\centering{2}                       & Investigación preliminar                          & \raggedleft{80} & 30-09-2024 & 25-10-2024 \\ \hline
		\centering{3}                       & Prototipado del proyecto                          & \raggedleft{30} & 28-10-2024 & 15-11-2024 \\ \hline
		\centering{4}                       & Desarrollo del firmware de los microcontroladores & \raggedleft{67} & 18-11-2024 & 20-12-2024 \\ \hline
		\multicolumn{2}{|r|}{Total Etapa 1} & \raggedleft{227}                                  & 20-08-2024      & 20-11-2024              \\ \hline
		\multicolumn{5}{|c|}{Etapa 2}                                                                                                       \\ \hline
		\rowcolor[HTML]{C0C0C0}
		WBS                                 & Tarea                                             & Horas           & Inicio     & Fin        \\ \hline
		\centering{4}                       & Desarrollo del firmware de los microcontroladores & \raggedleft{60} & 20-01-2025 & 14-02-2025 \\ \hline
		\centering{5}                       & Desarrollo del backend                            & \raggedleft{90} & 17-02-2025 & 21-03-2025 \\ \hline
		\centering{6}                       & Desarrollo del frontend                           & \raggedleft{93} & 24-03-2025 & 25-04-2025 \\ \hline
		\centering{7}                       & Pruebas y validación                              & \raggedleft{48} & 28-04-2025 & 16-05-2025 \\ \hline
		\centering{8}                       & Documentación del proyecto                        & \raggedleft{97} & 19-05-2025 & 20-06-2025 \\ \hline
		\multicolumn{2}{|r|}{Total Etapa 2} & \raggedleft{388}                                  & 20-01-2025      & 20-06-2025              \\ \hline
		\multicolumn{2}{|r|}{Totales}       & \raggedleft{615}                                  & 20-08-2024      & 20-06-2025              \\ \hline
	\end{tabularx}
	\caption{Lista de tareas del proyecto con fechas.}
	\label{tab:tabGantt}
\end{table}

Para el desarrollo de las actividades se tiene en cuenta las siguientes
consideraciones:

\begin{itemize}
	\item El proyecto se va a realizar en dos etapas. La primer etapa es en el año 2024 y la segunda en el año 2025.
	\item Etapa 1: Se estima una proyección de 12 horas por semana de dedicación en el
	      desarrollo del proyecto.
	\item El desarrollo del firmware de los microcontroladores se realiza en dos partes:
	      \begin{itemize}
		      \item Primera parte: Durante el año 2024 se realiza un avance estimado de 67 horas en
		            el desarrollo de esta actividad. Se estima avanzar hasta la Implementación del
		            firmware del nodo sensor de pH, CE y TDS.
		      \item Segunda parte: Durante el año 2024 se realiza un avance estimado de 60 horas en
		            el desarrollo de esta actividad.
	      \end{itemize}
	\item Etapa 2: Se estima una proyección de 20 horas por semana de dedicación en el
	      desarrollo del proyecto hasta la culminación del mismo.
\end{itemize}

\subsection{Diagramas}

Para visualizar el diagrama de Gantt de manera correcta, se procede a generar una figura correspondiente a cada tarea.
Al finalizar, se verá el diagrama completo de todas las actividades a ser desarrolladas en el proyecto.


En la figura \ref{fig:gantt1}, se puede visualizar el detalle de actividades referidas a la \textbf{tarea 1: Planificación del proyecto}.

% Diagrama de Gantt Planificación del proyecto
\begin{figure}[H]
		\begin{ganttchart}[
				time slot unit=day,
				time slot format=isodate,
				x unit=0.35cm, % Aumenta el ancho de las barras
				y unit title=0.7cm, % Aumenta la altura de los títulos
				y unit chart=0.7cm, % Aumenta la altura de las filas del gráfico
				milestone/.append style={xscale=4},
				vgrid,
				hgrid,
			]{2024-08-20}{2024-09-18}
			\gantttitlecalendar*{2024-08-20}{2024-09-18}{year} \\
			\gantttitlecalendar*{2024-08-20}{2024-09-18}{month} \\
			%\gantttitlecalendar*{2024-08-20}{2024-09-18}{day} \\
			\ganttbar{1.1 Elaboración del documento}{2024-08-20}{2024-09-17} \\
			\ganttbar{1.2 Diseño de la arquitectura}{2024-09-09}{2024-09-18} \\
		\end{ganttchart}
		\caption{Planificación del proyecto.}

	\label{fig:gantt1}
\end{figure}

En la figura \ref{fig:gantt2}, se puede visualizar el detalle de actividades referidas a la \textbf{tarea 2: Investigación preliminar}.

% Diagrama de Gantt Investigación preliminar
\begin{figure}[H]
		\begin{ganttchart}[
				time slot unit=day,
				time slot format=isodate,
				x unit=0.389cm, % Aumenta el ancho de las barras
				y unit title=0.7cm, % Aumenta la altura de los títulos
				y unit chart=0.7cm, % Aumenta la altura de las filas del gráfico
				milestone/.append style={xscale=4},
				vgrid,
				hgrid,
			]{2024-09-30}{2024-10-25}
			\gantttitlecalendar*{2024-09-30}{2024-10-25}{year} \\
			\gantttitlecalendar*{2024-09-30}{2024-10-25}{month} \\
			%\gantttitlecalendar*{2024-10-07}{2024-10-25}{day} \\
			\ganttbar{2.1 MQTT, HTTP y WebSockets}{2024-09-30}{2024-10-04} \\
			\ganttbar{2.2 Certificados TLS}{2024-10-07}{2024-10-09} \\
			\ganttbar{2.3 Microcontroladores}{2024-10-09}{2024-10-10} \\
			\ganttbar{2.4 Sensores y actuadores}{2024-10-11}{2024-10-15} \\
			\ganttbar{2.5 Base de datos}{2024-10-15}{2024-10-17} \\
			\ganttbar{2.6 Framework Backend}{2024-10-18}{2024-10-24} \\
			\ganttbar{2.7 Framework Fronted}{2024-10-18}{2024-10-24} \\
			\ganttbar{2.8 plataforma IoT}{2024-10-21}{2024-10-24} \\
			\ganttbar{2.9 Entorno de desarrollo}{2024-10-24}{2024-10-25} \\
		\end{ganttchart}
	\caption{Investigación preliminar.}
	\label{fig:gantt2}
\end{figure}

En la figura \ref{fig:gantt3}, se puede visualizar el detalle de actividades referidas a la \textbf{tarea 3: Prototipado del proyecto}.

% Diagrama de Gantt Prototipado del proyecto
\begin{figure}[H]
		\begin{ganttchart}[
				time slot unit=day,
				time slot format=isodate,
				x unit=0.53cm, % Aumenta el ancho de las barras
				y unit title=0.7cm, % Aumenta la altura de los títulos
				y unit chart=0.7cm, % Aumenta la altura de las filas del gráfico
				milestone/.append style={xscale=4},
				vgrid,
				hgrid,
			]{2024-10-28}{2024-11-15}
			\gantttitlecalendar*{2024-10-28}{2024-11-15}{year} \\
			\gantttitlecalendar*{2024-10-28}{2024-11-15}{month} \\
			%\gantttitlecalendar*{2024-08-20}{2024-09-18}{day} \\
			\ganttbar{3.1 Conexionado de componentes}{2024-10-28}{2024-11-10} \\
			\ganttbar{3.2 Integración de componentes}{2024-11-04}{2024-11-15} \\
		\end{ganttchart}
	\caption{Prototipado del proyecto.}
	\label{fig:gantt3}
\end{figure}

En la figura \ref{fig:gantt4.1}, se puede visualizar el detalle de actividades referidas a la primera parte
de la \textbf{tarea 4: Desarrollo del firmware de los microcontroladores}.

% Diagrama de Gantt Desarrollo del firmware de los microcontroladores (primera parte)
\begin{figure}[H]
		\begin{ganttchart}[
				time slot unit=day,
				time slot format=isodate,
				x unit=0.33cm, % Aumenta el ancho de las barras
				y unit title=0.7cm, % Aumenta la altura de los títulos
				y unit chart=0.7cm, % Aumenta la altura de las filas del gráfico
				milestone/.append style={xscale=4},
				vgrid,
				hgrid,
			]{2024-11-18}{2024-12-20}
			\gantttitlecalendar*{2024-11-18}{2024-12-20}{year} \\
			\gantttitlecalendar*{2024-11-18}{2024-12-20}{month} \\
			%\gantttitlecalendar*{2024-11-18}{2024-12-20}{day} \\
			\ganttbar{4.1 Configuraciones iniciales}{2024-11-18}{2024-11-18} \\
			\ganttbar{4.2 Implementación TLS}{2024-11-18}{2024-11-25} \\
			\ganttbar{4.3 Sensor Ambiental}{2024-11-25}{2024-12-04} \\
			\ganttbar{4.4 Sensor ($CO_2$)}{2024-12-05}{2024-12-10} \\
			\ganttbar{4.5 Sensor pH, CE y TDS}{2024-12-11}{2024-12-20} \\
		\end{ganttchart}
	\caption{Desarrollo del firmware de los microcontroladores (primera parte).}
	\label{fig:gantt4.1}
\end{figure}

En la figura \ref{fig:gantt4.2}, se puede visualizar el detalle de actividades referidas a la segunda parte
de la \textbf{tarea 4: Desarrollo del firmware de los microcontroladores}.

% Diagrama de Gantt Desarrollo del firmware de los microcontroladores (segunda parte)
\begin{figure}[H]
		\begin{ganttchart}[
				time slot unit=day,
				time slot format=isodate,
				x unit=0.42cm, % Aumenta el ancho de las barras
				y unit title=0.7cm, % Aumenta la altura de los títulos
				y unit chart=0.7cm, % Aumenta la altura de las filas del gráfico
				milestone/.append style={xscale=4},
				vgrid,
				hgrid,
			]{2025-01-20}{2025-02-14}
			\gantttitlecalendar*{2025-01-20}{2025-02-14}{year} \\
			\gantttitlecalendar*{2025-01-20}{2025-02-14}{month} \\
			%\gantttitlecalendar*{2024-11-18}{2024-12-20}{day} \\
			\ganttbar{4.6 Sensor solución nutritiva}{2025-01-20}{2025-01-27} \\
			\ganttbar{4.7 Sensor consumos}{2025-01-28}{2025-02-06} \\
			\ganttbar{4.8 Nodo actuador}{2025-02-07}{2025-02-14} \\

		\end{ganttchart}
	\caption{Desarrollo del firmware de los microcontroladores (segunda parte).}
	\label{fig:gantt4.2}
\end{figure}

En la figura \ref{fig:gantt5}, se puede visualizar el detalle de actividades referidas a la \textbf{tarea 5: Desarrollo del backend}.

% Diagrama de Gantt Desarrollo del backend
\begin{figure}[H]
		\begin{ganttchart}[
				time slot unit=day,
				time slot format=isodate,
				x unit=0.3cm, % Aumenta el ancho de las barras
				y unit title=0.7cm, % Aumenta la altura de los títulos
				y unit chart=0.7cm, % Aumenta la altura de las filas del gráfico
				milestone/.append style={xscale=4},
				vgrid,
				hgrid,
			]{2025-02-17}{2025-03-21}
			\gantttitlecalendar*{2025-02-17}{2025-03-21}{year} \\
			\gantttitlecalendar*{2025-02-17}{2025-03-21}{month} \\
			%\gantttitlecalendar*{2024-10-07}{2024-10-25}{day} \\
			\ganttbar{5.1 Configuración inicial}{2025-02-17}{2025-02-18} \\
			\ganttbar{5.2 Implementación TLS}{2025-02-18}{2025-02-20} \\
			\ganttbar{5.3 Broker MQTT}{2025-02-21}{2025-02-25} \\
			\ganttbar{5.4 Configuración JWT}{2025-02-25}{2025-02-27} \\
			\ganttbar{5.5 Base de datos}{2025-02-27}{2025-03-03} \\
			\ganttbar{5.6 Configuración WebSockets}{2025-03-04}{2025-03-06} \\
			\ganttbar{5.7 y 5.8 Endpoints usuarios}{2025-03-07}{2025-03-10} \\
			\ganttbar{5.9 y 5.10 Endpoints ambientes}{2025-03-10}{2025-03-11} \\
			\ganttbar{5.11 y 5.12 Endpoints sensores}{2025-03-11}{2025-03-13} \\
			\ganttbar{5.13 y 5.14 Endpoints actuadores}{2025-03-13}{2025-03-14} \\
			\ganttbar{5.15 y 5.16 Mediciones sensores}{2025-03-17}{2025-03-18} \\
			\ganttbar{5.17 y 5.18 Estados actuadores}{2025-03-18}{2025-03-19} \\
			\ganttbar{5.19 y 5.20 Parámetros sensores}{2025-03-19}{2025-03-20} \\
			\ganttbar{5.21 y 5.22 Parámetros actuadores}{2025-03-20}{2025-03-21} \\
		\end{ganttchart}
	\caption{Desarrollo del Backend.}
	\label{fig:gantt5}
\end{figure}


En la figura \ref{fig:gantt6}, se puede visualizar el detalle de actividades referidas a la \textbf{tarea 5: Desarrollo del frontend}.

% Diagrama de Gantt Desarrollo del frontend
\begin{figure}[H]
		\begin{ganttchart}[
				time slot unit=day,
				time slot format=isodate,
				x unit=0.29cm, % Aumenta el ancho de las barras
				y unit title=0.7cm, % Aumenta la altura de los títulos
				y unit chart=0.7cm, % Aumenta la altura de las filas del gráfico
				milestone/.append style={xscale=4},
				vgrid,
				hgrid,
			]{2025-03-24}{2025-04-25}
			\gantttitlecalendar*{2025-03-24}{2025-04-25}{year} \\
			\gantttitlecalendar*{2025-03-24}{2025-04-25}{month} \\
			%\gantttitlecalendar*{2025-03-24}{2025-04-25}{day} \\
			\ganttbar{6.1 Configuración inicial}{2025-03-24}{2025-03-26} \\
			\ganttbar{6.2 Implementación TLS}{2025-03-26}{2025-03-28} \\
			\ganttbar{6.3 Configuración JWT}{2025-03-31}{2025-04-02} \\
			\ganttbar{6.4 Configuración WebSockets}{2025-04-03}{2025-04-04} \\
			\ganttbar{6.5 y 6.6 Login de usuario}{2025-04-04}{2025-04-07} \\
			\ganttbar{6.7 y 6.8 Vistas de  usuarios}{2025-04-07}{2025-04-08} \\
			\ganttbar{6.9 y 6.10 Vistas de ambientes}{2025-04-08}{2025-04-09} \\
			\ganttbar{6.11 y 6.12 Vistas de sensores}{2025-04-09}{2025-04-10} \\
			\ganttbar{6.13 y 6.14 Vistas de actuadores}{2025-04-10}{2025-04-11} \\
			\ganttbar{6.15 y 6.16 Vista de mediciones}{2025-04-14}{2025-04-16} \\
			\ganttbar{6.17 y 6.18 Vista de estados}{2025-04-16}{2025-04-18} \\
			\ganttbar{6.19 y 6.20 Vista parámetro sensor}{2025-04-21}{2025-04-21} \\
			\ganttbar{6.21 y 6.22 Vista parámetro actuador}{2025-04-22}{2025-04-22} \\
			\ganttbar{6.23 y 6.24 Vista principal}{2025-04-23}{2025-04-25} \\
		\end{ganttchart}
	\caption{Desarrollo del Frontend.}
	\label{fig:gantt6}
\end{figure}

En la figura \ref{fig:gantt7}, se puede visualizar el detalle de actividades referidas a la \textbf{tarea 7: Pruebas y validación}.

% Diagrama de Gantt Pruebas y validación
\begin{figure}[H]
		\begin{ganttchart}[
				time slot unit=day,
				time slot format=isodate,
				x unit=0.5cm, % Aumenta el ancho de las barras
				y unit title=0.7cm, % Aumenta la altura de los títulos
				y unit chart=0.7cm, % Aumenta la altura de las filas del gráfico
				milestone/.append style={xscale=4},
				vgrid,
				hgrid,
			]{2025-04-28}{2025-05-16}
			\gantttitlecalendar*{2025-04-28}{2025-05-16}{year} \\
			\gantttitlecalendar*{2025-04-28}{2025-05-16}{month} \\
			%\gantttitlecalendar*{2025-04-28}{2025-05-16}{day} \\
			\ganttbar{7.1 Pruebas de los microcontroladores}{2025-04-28}{2025-05-02} \\
			\ganttbar{7.2 Pruebas Backend}{2025-05-05}{2025-05-09} \\
			\ganttbar{7.3 Pruebas Frontend}{2025-05-12}{2025-05-16} \\
		\end{ganttchart}
	\caption{Pruebas y validación.}
	\label{fig:gantt7}
\end{figure}

En la figura \ref{fig:gantt8}, se puede visualizar el detalle de actividades referidas a la \textbf{tarea 8: Documentación del proyecto}.

% Diagrama de Gantt Documentación del proyecto
\begin{figure}[H]
		\begin{ganttchart}[
				time slot unit=day,
				time slot format=isodate,
				x unit=0.36cm, % Aumenta el ancho de las barras
				y unit title=0.7cm, % Aumenta la altura de los títulos
				y unit chart=0.7cm, % Aumenta la altura de las filas del gráfico
				milestone/.append style={xscale=4},
				vgrid,
				hgrid,
			]{2025-05-19}{2025-06-20}
			\gantttitlecalendar*{2025-05-19}{2025-06-20}{year} \\
			\gantttitlecalendar*{2025-05-19}{2025-06-20}{month} \\
			%\gantttitlecalendar*{2025-05-19}{2025-06-20}{day} \\
			\ganttbar{8.1 Informe de avance}{2025-05-19}{2025-05-21} \\
			\ganttbar{8.2 Video del sistema}{2025-05-22}{2025-05-22} \\
			\ganttbar{8.3 Estructura memoria}{2025-05-22}{2025-05-23} \\
			\ganttbar{8.4 Escritura memoria}{2025-05-26}{2025-06-10} \\
			\ganttbar{8.5 Revisión memoria}{2025-06-14}{2025-06-18} \\
			\ganttbar{8.6 Presentación final}{2025-06-16}{2025-06-20} \\
		\end{ganttchart}
	\caption{Documentación del proyecto.}
	\label{fig:gantt8}
\end{figure}

Por último, en la figura \ref{fig:ganttGral}, se puede visualizar el total de las actividades a ser desarrolladas en  el \textbf{proyecto}.

% Diagrama de Gantt Documentación del proyecto
\begin{figure}[H]
		\begin{ganttchart}[
				time slot unit=day,
				time slot format=isodate,
				x unit=0.05cm, % Aumenta el ancho de las barras
				y unit title=0.39cm, % Aumenta la altura de los títulos
				y unit chart=0.39cm, % Aumenta la altura de las filas del gráfico
				milestone/.append style={xscale=4},
				vgrid,
				hgrid,
			]{2024-08-20}{2025-06-20}
			\gantttitlecalendar*{2024-08-20}{2025-06-20}{year} \\
			\gantttitlecalendar*{2024-08-20}{2025-06-20}{month} \\
			%\gantttitlecalendar*{2025-05-19}{2025-06-20}{day} \\
			\ganttbar{1.1}{2024-08-20}{2024-09-17} \\
			\ganttbar{1.2}{2024-09-09}{2024-09-18} \\

			\ganttbar{2.1}{2024-09-30}{2024-10-04} \\
			\ganttbar{2.2}{2024-10-07}{2024-10-09} \\
			\ganttbar{2.3}{2024-10-09}{2024-10-10} \\
			\ganttbar{2.4}{2024-10-11}{2024-10-15} \\
			\ganttbar{2.5}{2024-10-15}{2024-10-17} \\
			\ganttbar{2.6}{2024-10-18}{2024-10-24} \\
			\ganttbar{2.7}{2024-10-18}{2024-10-24} \\
			\ganttbar{2.8}{2024-10-21}{2024-10-24} \\
			\ganttbar{2.9}{2024-10-24}{2024-10-25} \\

			\ganttbar{3.1}{2024-10-28}{2024-11-10} \\
			\ganttbar{3.2}{2024-11-04}{2024-11-15} \\

			\ganttbar{4.1}{2024-11-18}{2024-11-18} \\
			\ganttbar{4.2}{2024-11-18}{2024-11-25} \\
			\ganttbar{4.3}{2024-11-25}{2024-12-04} \\
			\ganttbar{4.4}{2024-12-05}{2024-12-10} \\
			\ganttbar{4.5}{2024-12-11}{2024-12-20} \\
			\ganttbar{4.6}{2025-01-20}{2025-01-27} \\
			\ganttbar{4.7}{2025-01-28}{2025-02-06} \\
			\ganttbar{4.8}{2025-02-07}{2025-02-14} \\

			\ganttbar{5.1}{2025-02-17}{2025-02-18} \\
			\ganttbar{5.2}{2025-02-18}{2025-02-20} \\
			\ganttbar{5.3}{2025-02-21}{2025-02-25} \\
			\ganttbar{5.4}{2025-02-25}{2025-02-27} \\
			\ganttbar{5.5}{2025-02-27}{2025-03-03} \\
			\ganttbar{5.6}{2025-03-04}{2025-03-06} \\
			\ganttbar{5.7}{2025-03-07}{2025-03-10} \\
			\ganttbar{5.9}{2025-03-10}{2025-03-11} \\
			\ganttbar{5.11}{2025-03-11}{2025-03-13} \\
			\ganttbar{5.13}{2025-03-13}{2025-03-14} \\
			\ganttbar{5.15}{2025-03-17}{2025-03-18} \\
			\ganttbar{5.17}{2025-03-18}{2025-03-19} \\
			\ganttbar{5.19}{2025-03-19}{2025-03-20} \\
			\ganttbar{5.21}{2025-03-20}{2025-03-21} \\

			\ganttbar{6.1}{2025-03-24}{2025-03-26} \\
			\ganttbar{6.2}{2025-03-26}{2025-03-28} \\
			\ganttbar{6.3}{2025-03-31}{2025-04-02} \\
			\ganttbar{6.4}{2025-04-03}{2025-04-04} \\
			\ganttbar{6.5}{2025-04-04}{2025-04-07} \\
			\ganttbar{6.7}{2025-04-07}{2025-04-08} \\
			\ganttbar{6.9}{2025-04-08}{2025-04-09} \\
			\ganttbar{6.11}{2025-04-09}{2025-04-10} \\
			\ganttbar{6.13}{2025-04-10}{2025-04-11} \\
			\ganttbar{6.15}{2025-04-14}{2025-04-16} \\
			\ganttbar{6.17}{2025-04-16}{2025-04-18} \\
			\ganttbar{6.19}{2025-04-21}{2025-04-21} \\
			\ganttbar{6.21}{2025-04-22}{2025-04-22} \\
			\ganttbar{6.23}{2025-04-23}{2025-04-25} \\

			\ganttbar{7.1}{2025-04-28}{2025-05-02} \\
			\ganttbar{7.2}{2025-05-05}{2025-05-09} \\
			\ganttbar{7.3}{2025-05-12}{2025-05-16} \\

			\ganttbar{8.1}{2025-05-19}{2025-05-21} \\
			\ganttbar{8.2}{2025-05-22}{2025-05-22} \\
			\ganttbar{8.3}{2025-05-22}{2025-05-23} \\
			\ganttbar{8.4}{2025-05-26}{2025-06-10} \\
			\ganttbar{8.5}{2025-06-14}{2025-06-18} \\
			\ganttbar{8.6}{2025-06-16}{2025-06-20} \\
		\end{ganttchart}
	\caption{Diagrama de Gantt.}
	\label{fig:ganttGral}
\end{figure}