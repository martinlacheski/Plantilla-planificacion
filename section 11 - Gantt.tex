En la figura \ref{fig:gantt}, se muestra un ejemplo de diagrama de gantt
realizado con el paquete de \textit{pgfgantt}. En la plantilla pueden ver el
código que lo genera y usarlo de base para construir el propio.

Las fechas pueden ser calculadas utilizando alguna de las herramientas antes
citadas. Sin embargo, el siguiente ejemplo fue elaborado utilizando
\href{https://docs.google.com/spreadsheets/d/1fBz8NhSpc4tkkhz3KjJCbh1nR_ltDkfEcZi4tZXduqs}{esta
	hoja de cálculo}.

Es importante destacar que el ancho del diagrama estará dado por la longitud
del texto utilizado para las tareas (Ejemplo: tarea 1, tarea 2, etcétera) y el
valor \textit{x unit}. Para mejorar la apariencia del diagrama, es necesario
ajustar este valor y, quizás, acortar los nombres de las tareas.

% Diagrama de Gantt Planificación del proyecto
\begin{landscape}
	\begin{figure}[htpb]
		\begin{center}
			\begin{ganttchart}[
					time slot unit=day,
					time slot format=isodate,
					x unit=0.5cm, % Aumenta el ancho de las barras
					y unit title=0.7cm, % Aumenta la altura de los títulos
					y unit chart=0.7cm, % Aumenta la altura de las filas del gráfico
					milestone/.append style={xscale=4},
					vgrid,
					hgrid,
				]{2024-08-20}{2024-09-10}
				\gantttitlecalendar*{2024-08-20}{2024-09-10}{year} \\
				\gantttitlecalendar*{2024-08-20}{2024-09-10}{month} \\
				\gantttitlecalendar*{2024-08-20}{2024-09-10}{day} \\

				\ganttgroup{Planificación del proyecto}{2024-08-20}{2024-09-10} \\
				\ganttbar{Elaboración del documento de planificación}{2024-08-20}{2024-09-05} \\
				\ganttbar{Diseño de la arquitectura del proyecto}{2024-09-06}{2024-09-10} \\
			\end{ganttchart}
		\end{center}
		\caption{Diagrama de Gantt del Proyecto}
		\label{fig:gantt1}
	\end{figure}
\end{landscape}

% Diagrama de Gantt Investigación preliminar
\begin{landscape}
	\begin{figure}[htpb]
		\begin{center}
			\begin{ganttchart}[
					time slot unit=day,
					time slot format=isodate,
					x unit=0.5cm, % Aumenta el ancho de las barras
					y unit title=0.7cm, % Aumenta la altura de los títulos
					y unit chart=0.7cm, % Aumenta la altura de las filas del gráfico
					milestone/.append style={xscale=4},
					vgrid,
					hgrid,
				]{2024-09-11}{2024-10-13}
				\gantttitlecalendar*{2024-09-11}{2024-10-13}{year} \\
				\gantttitlecalendar*{2024-09-11}{2024-10-13}{month} \\
				\gantttitlecalendar*{2024-09-11}{2024-10-13}{day} \\

				\ganttgroup{Investigación preliminar}{2024-09-11}{2024-10-13} \\
				\ganttbar{Investigación de protocolos}{2024-09-11}{2024-09-16} \\
				\ganttbar{Investigación de  TLS}{2024-09-17}{2024-09-18} \\
				\ganttbar{Investigación de microcontroladores}{2024-09-19}{2024-09-20} \\
				\ganttbar{Investigación de sensores y actuadores}{2024-09-21}{2024-09-24} \\
				\ganttbar{Investigación de la base de datos}{2024-09-25}{2024-09-28} \\
				\ganttbar{Investigación del framework backend}{2024-09-29}{2024-10-02} \\
				\ganttbar{Investigación del framework frontend}{2024-10-03}{2024-10-06} \\
				\ganttbar{Investigación del servidor IoT}{2024-10-07}{2024-10-10} \\
				\ganttbar{Instalación del entorno de desarrollo}{2024-10-11}{2024-10-13} \\
			\end{ganttchart}
		\end{center}
		\caption{Diagrama de Gantt del Proyecto}
		\label{fig:gantt1}
	\end{figure}
\end{landscape}