En la tabla \ref{tab:tabGantt}, se puede visualizar un resumen del detalle de
la cantidad de horas, fecha de inicio y fecha de fin de cada grupo de tareas
estimadas a dedicar en el desarrollo del proyecto.

\begin{table}[ht]
	\begin{tabularx}{\linewidth}{|p{0.8cm}|p{8.33cm}|p{1cm}|p{1.8cm}|p{1.8cm}|}
		\hline
		%\rowcolor[HTML]{C0C0C0}
		%WBS                           & Tarea                                             & Horas           & Inicio     & Fin        \\ \hline
		%\textbf{ID} & \textbf{Tarea} & \textbf{Duración (días)} & \textbf{Fecha Inicio} & \textbf{Fecha Fin} \\ \hline
		\multicolumn{5}{|c|}{Etapa 1}                                                                                                       \\ \hline
		\rowcolor[HTML]{C0C0C0}
		WBS                                 & Tarea                                             & Horas           & Inicio     & Fin        \\ \hline
		\centering{1}                       & Planificación del proyecto                        & \raggedleft{50} & 20-08-2024 & 18-09-2024 \\ \hline
		\centering{2}                       & Investigación preliminar                          & \raggedleft{80} & 07-10-2024 & 25-10-2024 \\ \hline
		\centering{3}                       & Prototipado del proyecto                          & \raggedleft{30} & 28-10-2024 & 15-11-2024 \\ \hline
		\centering{4}                       & Desarrollo del firmware de los microcontroladores & \raggedleft{67} & 18-11-2024 & 20-12-2024 \\ \hline
		\multicolumn{2}{|r|}{Total Etapa 1} & \raggedleft{227}                                  & 20-08-2024      & 20-11-2024              \\ \hline
		\multicolumn{5}{|c|}{Etapa 2}                                                                                                       \\ \hline
		\rowcolor[HTML]{C0C0C0}
		WBS                                 & Tarea                                             & Horas           & Inicio     & Fin        \\ \hline
		\centering{4}                       & Desarrollo del firmware de los microcontroladores & \raggedleft{60} & 20-01-2025 & 14-02-2025 \\ \hline
		\centering{5}                       & Desarrollo del backend                            & \raggedleft{90} & 17-02-2025 & 21-03-2025 \\ \hline
		\centering{6}                       & Desarrollo del frontend                           & \raggedleft{93} & 24-03-2025 & 25-04-2025 \\ \hline
		\centering{7}                       & Pruebas y validación                              & \raggedleft{48} & 28-04-2025 & 16-05-2025 \\ \hline
		\centering{8}                       & Documentación del proyecto                        & \raggedleft{97} & 19-05-2025 & 20-06-2025 \\ \hline
		\multicolumn{2}{|r|}{Total Etapa 2} & \raggedleft{388}                                  & 20-01-2025      & 20-06-2025              \\ \hline
		\multicolumn{2}{|r|}{Totales}       & \raggedleft{615}                                  & 20-08-2024      & 20-06-2025              \\ \hline
	\end{tabularx}
	\caption{Lista de tareas del proyecto con fechas.}
	\label{tab:tabGantt}
\end{table}

Para el desarrollo de las actividades se tiene en cuenta las siguientes
consideraciones:

\begin{itemize}
	\item Etapa 1: Se estima una proyección de 15 horas por semana de dedicación en el
	      desarrollo del proyecto.
	\item El desarrollo del firmware de los microcontroladores se realiza en dos partes:
	      \begin{itemize}
		      \item Primera parte: Durante el año 2024 se realiza un avance estimado de 67 horas en
		            el desarrollo de esta actividad. Se estima culminar la Implementación del
		            firmware del nodo sensor de pH, CE y TDS
		      \item Segunda parte: Durante el año 2024 se realiza un avance estimado de 60 horas en
		            el desarrollo de esta actividad.
	      \end{itemize}
	\item Etapa 2: Se estima una proyección de 20 horas por semana de dedicación en el
	      desarrollo del proyecto hasta la culminación del mismo.
\end{itemize}