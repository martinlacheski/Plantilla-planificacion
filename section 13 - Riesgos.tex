\begin{consigna}{red}
	a) Identificación de los riesgos (al menos cinco) y estimación de sus consecuencias:

	Riesgo 1: detallar el riesgo (riesgo es algo que si ocurre altera los planes
	previstos de forma negativa)
	\begin{itemize}
		\item Severidad (S): mientras más severo, más alto es el número (usar números del 1
		      al 10).\\ Justificar el motivo por el cual se asigna determinado número de
		      severidad (S).
		\item Probabilidad de ocurrencia (O): mientras más probable, más alto es el número
		      (usar del 1 al 10).\\ Justificar el motivo por el cual se asigna determinado
		      número de (O).
	\end{itemize}

	Riesgo 2:
	\begin{itemize}
		\item Severidad (S): X.\\ Justificación...
		\item Ocurrencia (O): Y.\\ Justificación...
	\end{itemize}

	Riesgo 3:
	\begin{itemize}
		\item Severidad (S): X.\\ Justificación...
		\item Ocurrencia (O): Y.\\ Justificación...
	\end{itemize}

	b) Tabla de gestión de riesgos: (El RPN se calcula como RPN=SxO)

	\begin{table}[htpb]
		\centering
		\begin{tabularx}{\linewidth}{@{}|X|c|c|c|c|c|c|@{}}
			\hline
			\rowcolor[HTML]{C0C0C0}
			Riesgo & S & O & RPN & S* & O* & RPN* \\ \hline
			       &   &   &     &    &    &      \\ \hline
			       &   &   &     &    &    &      \\ \hline
			       &   &   &     &    &    &      \\ \hline
			       &   &   &     &    &    &      \\ \hline
			       &   &   &     &    &    &      \\ \hline
		\end{tabularx}%
	\end{table}

	Criterio adoptado:

	Se tomarán medidas de mitigación en los riesgos cuyos números de RPN sean
	mayores a...

	Nota: los valores marcados con (*) en la tabla corresponden luego de haber
	aplicado la mitigación.

	c) Plan de mitigación de los riesgos que originalmente excedían el RPN máximo
	establecido:

	Riesgo 1: plan de mitigación (si por el RPN fuera necesario elaborar un plan de
	mitigación). Nueva asignación de S y O, con su respectiva justificación:
	\begin{itemize}
		\item Severidad (S*): mientras más severo, más alto es el número (usar números del 1
		      al 10). Justificar el motivo por el cual se asigna determinado número de
		      severidad (S).
		\item Probabilidad de ocurrencia (O*): mientras más probable, más alto es el número
		      (usar del 1 al 10). Justificar el motivo por el cual se asigna determinado
		      número de (O).
	\end{itemize}

	Riesgo 2: plan de mitigación (si por el RPN fuera necesario elaborar un plan de
	mitigación).

	Riesgo 3: plan de mitigación (si por el RPN fuera necesario elaborar un plan de
	mitigación).

\end{consigna}