\textbf{Definición de riesgos.}

Riesgo 1 - Imposibilidad de cumplir con los plazos establecidos.
\begin{itemize}
	\item Severidad (S): 9.\\ Un retraso en el cronograma afectaría la entrega final del
	      proyecto y podría comprometer su éxito.
	\item Ocurrencia (O): 5.\\ El responsable tiene múltiples compromisos laborales,
	      académicos y familiares, lo que limita el tiempo disponible para dedicarse al
	      proyecto de forma continua.
\end{itemize}

Riesgo 2 - Retraso en la disponibilidad de los componentes para desarrollar el
prototipo.
\begin{itemize}
	\item Severidad (S): 7.\\ La falta de materiales en el tiempo establecido podría
	      retrasar la construcción, programación y prueba del prototipo, lo que afectaría
	      los plazos establecidos para completar el proyecto.
	\item Ocurrencia (O): 5.\\ Existe una posibilidad moderada de que los proveedores no
	      entreguen los materiales a tiempo, por demoras logísticas o problemas de
	      stock, debido a la dependencia de insumos específicos para el prototipo.
\end{itemize}

Riesgo 3 - Selección inadecuada de actuadores, sensores y microcontroladores.
\begin{itemize}
	\item Severidad (S): 8.\\ La elección incorrecta de estos componentes puede generar
	      fallos en el funcionamiento o necesidad de adquirir nuevos dispositivos, lo que
	      causaría retrasos significativos en el desarrollo y pruebas del proyecto,
	      además de incrementar los costos.
	\item Ocurrencia (O): 4.\\ Dado que el proyecto implica el uso de nuevas tecnologías
	      y el responsable no posee experiencia con algunos de estos componentes, existe
	      un riesgo moderado de que la selección inicial no sea la más adecuada.
\end{itemize}

Riesgo 4 - Falta de conocimientos adecuados para el desarrollo del proyecto.
\begin{itemize}
	\item Severidad (S): 7.\\ La falta de experiencia puede llevar a errores en el diseño
	      y desarrollo, aumentando el tiempo necesario para completar el proyecto y
	      potencialmente afectar la calidad y funcionalidad del sistema.
	\item Ocurrencia (O): 6.\\ Aunque es posible adquirir los conocimientos necesarios
	      mediante formación y consulta, existe una probabilidad moderada de enfrentar
	      dificultades debido a la curva de aprendizaje y posibles dificultades técnicas.
\end{itemize}

Riesgo 5 - Cambio de trabajo del responsable del proyecto.
\begin{itemize}
	\item Severidad (S): 9.\\ El cambio de trabajo del único responsable del proyecto
	      puede tener un impacto severo en el progreso, debido a que detendría todo el
	      desarrollo y requeriría la búsqueda de un reemplazo o una reestructuración
	      completa del proyecto. Esto puede causar retrasos muy significativos y posibles
	      problemas de continuidad.
	\item Ocurrencia (O): 4.\\ Aunque el cambio de trabajo es una posibilidad real, la
	      probabilidad de que ocurra durante el proyecto puede ser moderada.
\end{itemize}

\pagebreak

\textbf{Tabla de gestión de riesgos.}

El RPN se calcula como \textit{RPN = S x O}

\begin{table}[H]
	\centering
	\begin{tabularx}{\linewidth}{@{}|X|c|c|c|c|c|c|@{}}
		\hline
		\rowcolor[HTML]{C0C0C0}
		Riesgo                                                & S & O & RPN & S* & O* & RPN* \\ \hline
		Imposibilidad de cumplir con los plazos establecidos. & 9 & 5 & 45  & 9  & 4  & 36   \\ \hline
		Retraso en la disponibilidad de los componentes para
		desarrollar el prototipo.                             & 7 & 5 & 35  & -  & -  & -    \\ \hline
		Selección inadecuada de actuadores, sensores y
		microcontroladores.                                   & 8 & 4 & 32  & -  & -  & -    \\ \hline
		Falta de conocimientos adecuados para el desarrollo del
		proyecto.                                             & 7 & 6 & 42  & 7  & 5  & 35   \\ \hline
		Cambio de trabajo del responsable del proyecto.       & 9 & 4 & 36  & -  & -  & -    \\ \hline
	\end{tabularx}%
\end{table}

\textbf{Criterio adoptado:}

Se tomarán medidas de mitigación en los riesgos cuyos números de RPN sean
mayores a 40.

Nota: los valores marcados con (*) en la tabla corresponden luego de haber
aplicado la mitigación.

\textbf{Plan de mitigación de los riesgos que originalmente excedían el RPN máximo
	establecido.}

Riesgo 1 - Imposibilidad de cumplir con los plazos establecidos.
\begin{itemize}
	\item Planes de mitigación:\\
	      \begin{itemize}
		      \item Establecer hitos intermedios:\\ Dividir el proyecto en fases más pequeñas con
		            fechas límite claras. Esto permitirá monitorear el progreso de manera más
		            efectiva y detectar posibles retrasos a tiempo.
		      \item Implementar un sistema de seguimiento y control:\\ Utilizar herramientas de
		            gestión de proyectos para monitorear el avance del proyecto en tiempo real y
		            detectar desviaciones o retrasos del plan.
	      \end{itemize}
	\item Severidad (S*): 9.\\ La severidad se mantiene constante, pero baja la probabilidad de ocurrencia.
	\item Probabilidad de ocurrencia (O*): 4.\\ Aunque sigue siendo moderada, debido a que
	      el responsable del proyecto sigue teniendo múltiples compromisos, la
	      probabilidad se ajusta para reflejar una disminución considerable en el riesgo,
	      gracias a las medidas proactivas.
\end{itemize}

\pagebreak

Riesgo 4 - Falta de conocimientos adecuados para el desarrollo del proyecto.
\begin{itemize}
	\item Planes de mitigación:\\
	      \begin{itemize}
		      \item Formación continua y capacitación:\\ Realizar cursos especializados en las
		            áreas clave del proyecto, como desarrollo de sistemas embebidos, backend y
		            frontend para asegurar un conocimiento profundo y estructurado.
		      \item Asesoramiento y mentoría:\\ Contactar a expertos o mentores en el campo para
		            recibir orientación y consejos técnicos. Unirse a comunidades o foros
		            especializados para intercambiar experiencias y obtener ayuda con problemas
		            específicos.
	      \end{itemize}
	\item Severidad (S*): 7.\\ La severidad se mantiene constante, pero baja la probabilidad de ocurrencia.
	\item Probabilidad de ocurrencia (O*): 5.\\ La probabilidad de enfrentar dificultades
	      debido a la curva de aprendizaje se reduce con la formación continua y la
	      asesoría. Sin embargo, la curva de aprendizaje y los problemas técnicos pueden
	      aún presentar desafíos, pero la probabilidad de que estos problemas afecten
	      significativamente el proyecto disminuye con un enfoque proactivo.
\end{itemize}