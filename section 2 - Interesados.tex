En la tabla \ref{tab:interesados}, se pueden identificar los participantes interesados del proyecto.


\begin{table}[ht]
	\begin{tabularx}{\linewidth}{|p{2.15cm}|p{5.8cm}|p{2.85cm}|p{3.4cm}|}
		\hline
		\rowcolor[HTML]{C0C0C0}
		Rol                           & Nombre y Apellido                                                                              & Organización    & Puesto            \\ \hline
		Cliente                       & \clientename                                                                                   & \empclientename & Director Proyecto \\ \hline
		Responsable                   & \authorname                                                                                    & FIUBA           & Alumno            \\ \hline
		\multirow{2}{*}{Orientadores} & \supname                                                                                       & \pertesupname   & Director          \\
		                              & \cosupname                                                                                     & \pertecosupname & Codirectora       \\ \hline
		Opositores                    & \multicolumn{3}{l|}{Empresas que ya ofrecen soluciones similares en el mercado.}                                                     \\ \hline
		Usuario final                 & \multicolumn{3}{l|}{Organizaciones interesadas en la automatización de cultivos en invernaderos.}                                       \\ \hline
	\end{tabularx}
	\caption{Identificación de los interesados.}
	\label{tab:interesados}
\end{table}

\begin{itemize}
	\item Cliente: El \clientename\hspace{1px} es especialista en cultivos y gestión forestal, con amplia experiencia en
	      cultivos hidropónicos. Va a colaborar con la definición de los requerimientos y el seguimiento del proyecto.
	\item Orientadores:
	      \begin{itemize}
		      \item El Director \supname\hspace{1px} es experto en la temática y guiará con la
		            implementación de los protocolos y herramientas del proyecto.
		      \item La Codirectora \cosupname , experta en Sistemas de Información, ayudará con el
		            seguimiento metodológico, para garantizar una gestión rigurosa y efectiva del
		            desarrollo del trabajo.
	      \end{itemize}
\end{itemize}

