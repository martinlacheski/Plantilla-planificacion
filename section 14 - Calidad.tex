\begin{consigna}{red}
	Elija al menos diez requerimientos que a su criterio sean los más importantes/críticos/que aportan más valor y para cada uno de ellos indique las acciones de verificación y validación que permitan asegurar su cumplimiento.

	\begin{itemize}
		\item Req \#1: copiar acá el requerimiento con su correspondiente número.

		      \begin{itemize}
			      \item Verificación para confirmar si se cumplió con lo requerido antes de mostrar el
			            sistema al cliente. Detallar.
			      \item Validación con el cliente para confirmar que está de acuerdo en que se cumplió
			            con lo requerido. Detallar.
		      \end{itemize}

	\end{itemize}

	Tener en cuenta que en este contexto se pueden mencionar simulaciones,
	cálculos, revisión de hojas de datos, consulta con expertos, mediciones, etc.

	Las acciones de verificación suelen considerar al entregable como ``caja
	blanca'', es decir se conoce en profundidad su funcionamiento interno.

	En cambio, las acciones de validación suelen considerar al entregable como
	``caja negra'', es decir, que no se conocen los detalles de su funcionamiento
	interno.

\end{consigna}