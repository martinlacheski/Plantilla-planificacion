Listado de requerimiento críticos con sus acciones de verificación y
validación:

\begin{itemize}
	\item Requerimientos asociados a los sensores y actuadores:
	      \begin{itemize}
		      \item Req \#1:\\ 1.2. Cada nodo deberá implementar certificados TLS.
		            \begin{itemize}
			            \item Verificación: Se realizarán pruebas para asegurar que la comunicación entre los
			                  nodos y el servidor es encriptada y segura.
			            \item Validación: Se comprobará que todas las conexiones con el microcontrolador
			                  utilicen certificados TLS.
		            \end{itemize}
		      \item Req \#2: \\ 1.5: Cada nodo deberá permitir configurar de manera remota el
		            tiempo para enviar los datos recolectados.
		            \begin{itemize}
			            \item Verificación: Se realizarán pruebas funcionales de la interfaz de configuración
			                  para ajustar el tiempo de envío de datos.
			            \item Validación: Se comprobará que se realizan las lecturas de los nodos en el
			                  tiempo solicitado.
		            \end{itemize}
		      \item Req \#3: \\ 1.6: Cada nodo deberá enviar al servidor IoT los datos
		            recolectados.
		            \begin{itemize}
			            \item Verificación: Se realizarán pruebas unitarias para verificar que los datos
			                  enviados por cada sensor son correctos y completos.
			            \item Validación: se realizarán mediciones de la magnitud correspondiente y se
			                  verificarán los datos obtenidos.
		            \end{itemize}
		      \item Req \#4: \\ 2.6: Cada nodo actuador deberá permitir configurar de manera remota
		            los parámetros del actuador y cada canal.
		            \begin{itemize}
			            \item Verificación: Se realizarán pruebas funcionales de la interfaz de configuración
			                  de los actuadores y sus canales.
			            \item Validación: Se comprobará que se activen y controlan los canales del actuador y
			                  se verificará su funcionamiento.
		            \end{itemize}
	      \end{itemize}

	\item Requerimientos asociados al Frontend:
	      \begin{itemize}
		      \item Req \#5:\\ 4.1: La interfaz deberá ser intuitiva y accesible desde dispositivos
		            móviles y de escritorio.
		            \begin{itemize}
			            \item Verificación: Se realizarán pruebas de usabilidad y compatibilidad en
			                  diferentes dispositivos (móviles, tablets, portátiles y computadoras de
			                  escritorio) para asegurar que la aplicación se visualice correctamente, se
			                  verificará la responsividad y el funcionamiento adecuado de cada componente.
			            \item Validación: Se comprobará que todas las secciones de la aplicación son fáciles
			                  de usar y se visualizan correctamente en diversos dispositivos, para confirmar
			                  que la experiencia de usuario sea la adecuada.
		            \end{itemize}
		      \item Req \#6: \\ 4.6: Deberá permitir visualizar en tiempo real los datos recibidos
		            de los sensores y actuadores.
		            \begin{itemize}
			            \item Verificación: Se llevarán a cabo pruebas de funcionalidad para asegurar que la
			                  interfaz muestra los datos en tiempo real de manera precisa y continua.
			            \item Validación: Se solicitará feedback a los usuarios finales para verificar que la
			                  visualización de los datos en tiempo real es clara y concisa.
		            \end{itemize}
		      \item Req \#7: \\ 4.7: Deberá permitir enviar comandos y configuraciones a los
		            sensores y actuadores.
		            \begin{itemize}
			            \item Verificación: Se realizarán pruebas funcionales de envío de comandos desde la
			                  interfaz y confirmación de la ejecución correcta.
			            \item Validación: Se realizarán pruebas de campo para confirmar que los comandos se
			                  ejecutan de manera correcta.
		            \end{itemize}
		      \item Req \#8: \\ 4.8: Deberá permitir la visualización de los datos históricos de
		            los sensores y actuadores.
		            \begin{itemize}
			            \item Verificación: Se realizarán pruebas para asegurar que la interfaz muestre
			                  correctamente los datos históricos de sensores y actuadores, comprobar que los
			                  filtros de búsqueda, la navegación entre fechas y la presentación de los datos
			                  funcionen adecuadamente.
			            \item Validación: Se solicitará al cliente y usuarios finales que revisen la interfaz
			                  y proporcionen retroalimentación sobre la utilidad, accesibilidad y claridad de
			                  la visualización de los datos históricos.
		            \end{itemize}
		      \item Req \#9: \\ 4.9: Deberá permitir la visualización en tiempo real y el control
		            de los sensores y actuadores a través de un tablero interactivo.
		            \begin{itemize}
			            \item Verificación: Se desarrollarán diferentes tipos de visualizaciones (como
			                  gráficos, tablas, etc.) en el tablero interactivo para asegurar que
			                  proporcionen información clara y efectiva. Se evaluará si permiten una
			                  visualización y control eficiente de los sensores y actuadores.
			            \item Validación: Se recogerá retroalimentación de clientes y usuarios finales para
			                  confirmar que el tablero interactivo es funcional, intuitivo y proporciona una
			                  visión general clara del sistema.
		            \end{itemize}
	      \end{itemize}

	\item Requerimientos asociados al Backend y la API:
	      \begin{itemize}
		      \item Req \#10:\\ 5.1: Deberá permitir conexiones seguras mediante TLS.
		            \begin{itemize}
			            \item Verificación: Se realizarán pruebas de seguridad y revisión de las bibliotecas
			                  utilizadas para asegurar que la comunicación es encriptada y segura.
			            \item Validación: Se comprobará que todas las conexiones utilicen TLS a través de la
			                  ejecución de requests a todos los endpoints disponibles.
		            \end{itemize}
		      \item Req \#11: \\ 5.2: Deberá poder implementar JWT (JSON Web Token) para poder
		            propagar entre dos partes, y de forma segura, la identidad de un determinado
		            usuario.
		            \begin{itemize}
			            \item Verificación: Se realizarán pruebas de integración para confirmar que el
			                  sistema puede generar y validar tokens JWT correctamente. Se verificarán
			                  aspectos como la encriptación del token, su expiración, y la transmisión de la
			                  identidad del usuario entre el cliente y el servidor. También se revisará que
			                  el proceso de autenticación y autorización basado en JWT funciona según lo
			                  especificado.
			            \item Validación: Se comprobará que el sistema proporciona una autenticación segura y
			                  que el manejo de los tokens JWT cumple con los requisitos de seguridad y
			                  funcionalidad esperados.
		            \end{itemize}
		      \item Req \#12: \\ 5.3: Deberá soportar los métodos HTTP para realizar operaciones
		            CRUD y visualizar los reportes, WebSockets para la visualización en tiempo real
		            de los datos y MQTT para la interacción con los sensores y actuadores.
		            \begin{itemize}
			            \item Verificación: Se realizarán pruebas de funcionalidad para asegurar que los
			                  métodos HTTP (GET, POST, PUT, DELETE) funcionan correctamente para las
			                  operaciones CRUD y la generación de reportes. Se comprobará la implementación
			                  de WebSockets para asegurar que la visualización en tiempo real de los datos es
			                  precisa. Además, se evaluará la comunicación MQTT para verificar que los
			                  mensajes se envían y reciben adecuadamente entre el servidor y los
			                  sensores/actuadores.
			            \item Validación: Se comprobará que los métodos HTTP, WebSockets y MQTT cumplen con
			                  las expectativas en términos de desempeño y funcionalidad.
		            \end{itemize}
		      \item Req \#13: \\ 5.8: Deberá poder persistir la información histórica de las
		            mediciones de los sensores.
		            \begin{itemize}
			            \item Verificación: Se realizarán pruebas de almacenamiento y recuperación de datos
			                  para verificar que la información histórica de las mediciones de los sensores
			                  se guarda correctamente en la base de datos.
			            \item Validación: se comprobará la persistencia de la información histórica es
			                  adecuada y accesible para consultas futuras.
		            \end{itemize}
	      \end{itemize}
\end{itemize}

\pagebreak

