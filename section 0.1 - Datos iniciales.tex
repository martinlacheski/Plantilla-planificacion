% El títulos de la memoria, se usa en la carátula y se puede usar el cualquier lugar del documento con el comando \ttitle
\titulo{Sistema de monitoreo y gestión remota del clima en invernaderos} 

% Nombre del posgrado, se usa en la carátula y se puede usar el cualquier lugar del documento con el comando \degreename
%\posgrado{Carrera de Especialización en Sistemas Embebidos} 
\posgrado{Carrera de Especialización en Internet de las Cosas} 
%\posgrado{Carrera de Especialización en Inteligencia Artificial}
%\posgrado{Maestría en Sistemas Embebidos} 
%\posgrado{Maestría en Internet de las cosas}
% IMPORTANTE: no omitir titulaciones ni tildación en los nombres, también se recomienda escribir los nombres completos (tal cual los tienen en su documento)
% Tu nombre, se puede usar el cualquier lugar del documento con el comando \authorname
\autor{Lic. Martín Anibal Lacheski}

% El nombre del director y co-director, se puede usar el cualquier lugar del documento con el comando \supname y \cosupname y \pertesupname y \pertecosupname
\director{Mg. Lic. Leopoldo Alfredo Zimperz}
\pertenenciaDirector{FIUBA} 
\codirector{Dra. Lic. Nancy Beatriz Ganz} % para que aparezca en la portada se debe descomentar la opción codirector en los parámetros de documentclass
\pertenenciaCoDirector{FCEQyN-UNaM}

% Nombre del cliente, quien va a aprobar los resultados del proyecto, se puede usar con el comando \clientename y \empclientename
\cliente{Ing. Facundo Ramos Hentz}
\empresaCliente{FCF-UNaM}

\horasProyecto{615 }
\costoProyecto{$X.XXX.XXX }

\fechaINICIO{20 de agosto de 2024}		%Fecha de inicio de la cursada de GdP \fechaInicioName
\fechaFINALPlan{8 de octubre de 2024} 	%Fecha de final de cursada de GdP
\fechaFINALTrabajo{en el mes de junio de 2025}	%Fecha de defensa pública del trabajo final