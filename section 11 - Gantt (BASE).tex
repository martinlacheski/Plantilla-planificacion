\begin{consigna}{red}
	Existen muchos programas y recursos \textit{online} para hacer diagramas de Gantt, entre los cuales destacamos:

	\begin{itemize}
		\item Planner
		\item GanttProject
		\item Trello + \textit{plugins}. En el siguiente link hay un tutorial oficial: \\
		      \url{https://blog.trello.com/es/diagrama-de-gantt-de-un-proyecto}
		\item Creately, herramienta online colaborativa.
		      \\\url{https://creately.com/diagram/example/ieb3p3ml/LaTeX}
		\item Se puede hacer en latex con el paquete \textit{pgfgantt}\\
		      \url{http://ctan.dcc.uchile.cl/graphics/pgf/contrib/pgfgantt/pgfgantt.pdf}
	\end{itemize}

	Pegar acá una captura de pantalla del diagrama de Gantt, cuidando que la letra
	sea suficientemente grande como para ser legible. Si el diagrama queda
	demasiado ancho, se puede pegar primero la ``tabla'' del Gantt y luego pegar la
	parte del diagrama de barras del diagrama de Gantt.

	Configurar el software para que en la parte de la tabla muestre los códigos del
	EDT (WBS).\\ Configurar el software para que al lado de cada barra muestre el
	nombre de cada tarea.\\ Revisar que la fecha de finalización coincida con lo
	indicado en el Acta Constitutiva.

	En la figura \ref{fig:gantt}, se muestra un ejemplo de diagrama de gantt
	realizado con el paquete de \textit{pgfgantt}. En la plantilla pueden ver el
	código que lo genera y usarlo de base para construir el propio.

	Las fechas pueden ser calculadas utilizando alguna de las herramientas antes
	citadas. Sin embargo, el siguiente ejemplo fue elaborado utilizando
	\href{https://docs.google.com/spreadsheets/d/1fBz8NhSpc4tkkhz3KjJCbh1nR_ltDkfEcZi4tZXduqs}{esta
		hoja de cálculo}.

	Es importante destacar que el ancho del diagrama estará dado por la longitud
	del texto utilizado para las tareas (Ejemplo: tarea 1, tarea 2, etcétera) y el
	valor \textit{x unit}. Para mejorar la apariencia del diagrama, es necesario
	ajustar este valor y, quizás, acortar los nombres de las tareas.

	\begin{landscape}
		\begin{figure}[htpb]
			\begin{center}
				\begin{ganttchart}[
						time slot unit=day,
						time slot format=isodate,
						x unit=0.3cm, % Ajusta el ancho de cada día
						y unit title=0.9cm, % Aumenta la altura de los títulos
						y unit chart=0.8cm, % Aumenta la altura de las filas del gráfico
						milestone/.append style={xscale=4},
						vgrid,
						hgrid,
						compress calendar % Permite ajustar el ancho de los días
						% today=2024-08-20, % Desactiva la línea de TODAY quitando o comentando esta línea
					]{2024-08-20}{2024-09-10}
					\gantttitlecalendar*{2024-08-20}{2024-09-10}{year} \\
					\gantttitlecalendar*{2024-08-20}{2024-09-10}{month} \\
					\gantttitlecalendar*{2024-08-20}{2024-09-10}{day} \\ % Agrega los días
	
					\ganttgroup{Planificación del proyecto}{2024-08-20}{2024-09-10} \\
					\ganttbar{Elaboración del documento de planificación}{2024-08-20}{2024-09-05} \\
					\ganttbar{Diseño de la arquitectura del proyecto}{2024-09-06}{2024-09-10} \\
	
				\end{ganttchart}
			\end{center}
			\caption{Diagrama de Gantt del Proyecto}
			\label{fig:gantt}
		\end{figure}
	\end{landscape}

	\begin{landscape}
		\begin{figure}[htpb]
			\centering
			\includegraphics[height=.85\textheight]{./Figuras/Gantt-2.png}
			\caption{Ejemplo de diagrama de Gantt (apaisado).} %Modificar este título acorde.
			\label{fig:diagGantt}
		\end{figure}

	\end{landscape}

\end{consigna}




\begin{landscape}
	\begin{figure}[htpb]
		\begin{center}
			\begin{ganttchart}[
					time slot unit=day,
					time slot format=isodate,
					x unit=0.038cm,
					y unit title=0.7cm,
					y unit chart=0.6cm,
					milestone/.append style={xscale=4},
					vgrid,
					hgrid,
					today=2024-08-20]{2024-10-21}{2024-11-10}
				\gantttitlecalendar*{2024-10-21}{2024-11-10}{year} \\
				\gantttitlecalendar*{2024-10-21}{2024-11-10}{month} \\

				\ganttgroup{Prototipado del proyecto}{2024-10-21}{2024-11-10} \\
				\ganttbar{Configuración y conexionado de sensores y actuadores}{2024-10-21}{2024-10-31} \\
				\ganttbar{Integración de componentes y microcontroladores}{2024-11-01}{2024-11-10} \\
			\end{ganttchart}
		\end{center}
		\caption{Diagrama de Gantt del Proyecto}
		\label{fig:gantt3}
	\end{figure}
\end{landscape}

\begin{landscape}
	\begin{figure}[htpb]
		\begin{center}
			\begin{ganttchart}[
					time slot unit=day,
					time slot format=isodate,
					x unit=0.038cm,
					y unit title=0.7cm,
					y unit chart=0.6cm,
					milestone/.append style={xscale=4},
					vgrid,
					hgrid,
					today=2024-08-20
				]{2024-08-20}{2025-06-30}
				\gantttitlecalendar*{2024-11-11}{2025-02-20}{year} \\
				\gantttitlecalendar*{2024-11-11}{2025-02-20}{month} \\

				\ganttgroup{Desarrollo del firmware}{2024-11-11}{2025-01-10} \\
				\ganttbar{Configuración inicial de los microcontroladores}{2024-11-11}{2024-11-12} \\
				\ganttbar{Implementación y configuración de certificados TLS}{2024-11-13}{2024-11-22} \\
				\ganttbar{Firmware nodo sensor de temperatura, humedad, presión y luz}{2024-11-23}{2024-12-07} \\
				\ganttbar{Firmware nodo sensor de dióxido de carbono}{2024-12-08}{2024-12-22} \\
				\ganttbar{Firmware nodo sensor de pH, CE y TDS}{2024-12-23}{2025-01-06} \\
				\ganttbar{Firmware nodo sensor de solución nutritiva}{2025-01-07}{2025-01-21} \\
				\ganttbar{Firmware nodo sensor de agua, nutrientes y energía eléctrica}{2025-01-22}{2025-02-05} \\
				\ganttbar{Firmware nodo actuador}{2025-02-06}{2025-02-20} \\

			\end{ganttchart}
		\end{center}
		\caption{Diagrama de Gantt del Proyecto}
		\label{fig:gantt}
	\end{figure}
\end{landscape}

\begin{landscape}
	\begin{figure}[htpb]
		\begin{center}
			\begin{ganttchart}[
					time slot unit=day,
					time slot format=isodate,
					x unit=0.038cm,
					y unit title=0.7cm,
					y unit chart=0.6cm,
					milestone/.append style={xscale=4},
					vgrid,
					hgrid,
					today=2024-08-20
				]{2024-08-20}{2025-06-30}
				\gantttitlecalendar*{2025-02-21}{2025-04-10}{year} \\
				\gantttitlecalendar*{2025-02-21}{2025-05-31}{month} \\

				\ganttgroup{Desarrollo del backend}{2025-02-21}{2025-04-10} \\
				\ganttbar{Configuración inicial del entorno}{2025-02-21}{2025-02-28} \\
				\ganttbar{Implementación del broker MQTT}{2025-03-01}{2025-03-10} \\
				\ganttbar{Configuración de certificados TLS y JWT}{2025-03-11}{2025-03-20} \\
				\ganttbar{Configuración y prueba de la base de datos}{2025-03-21}{2025-03-30} \\
				\ganttbar{Implementación de WebSockets y endpoints}{2025-03-31}{2025-04-10} \\

				\ganttgroup{Desarrollo del frontend}{2025-04-11}{2025-05-31} \\
				\ganttbar{Configuración del entorno y TLS}{2025-04-11}{2025-04-20} \\
				\ganttbar{Implementación de JWT y WebSockets}{2025-04-21}{2025-05-01} \\
				\ganttbar{Desarrollo de interfaces de usuario}{2025-05-02}{2025-05-31} \\
			\end{ganttchart}
		\end{center}
		\caption{Diagrama de Gantt del Proyecto}
		\label{fig:gantt}
	\end{figure}
\end{landscape}

\begin{landscape}
	\begin{figure}[htpb]
		\begin{center}
			\begin{ganttchart}[
					time slot unit=day,
					time slot format=isodate,
					x unit=0.038cm,
					y unit title=0.7cm,
					y unit chart=0.6cm,
					milestone/.append style={xscale=4},
					vgrid,
					hgrid,
					today=2024-08-20
				]{2024-08-20}{2025-06-30}
				\gantttitlecalendar*{2024-06-01}{2025-06-30}{year} \\
				\gantttitlecalendar*{2024-06-01}{2025-06-30}{month} \\

				\ganttgroup{Pruebas y validación}{2025-06-01}{2025-06-15} \\
				\ganttbar{Pruebas de microcontroladores y firmware}{2025-06-01}{2025-06-05} \\
				\ganttbar{Pruebas de backend y frontend}{2025-06-06}{2025-06-10} \\
				\ganttbar{Pruebas de seguridad y rendimiento}{2025-06-11}{2025-06-15} \\

				\ganttgroup{Documentación del proyecto}{2025-06-16}{2025-06-30} \\
				\ganttbar{Informe de avance y revisión}{2025-06-16}{2025-06-20} \\
				\ganttbar{Video demostración y presentación final}{2025-06-21}{2025-06-30} \\
			\end{ganttchart}
		\end{center}
		\caption{Diagrama de Gantt del Proyecto}
		\label{fig:gantt}
	\end{figure}
\end{landscape}