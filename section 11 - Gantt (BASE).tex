\begin{consigna}{red}
	Existen muchos programas y recursos \textit{online} para hacer diagramas de Gantt, entre los cuales destacamos:

	\begin{itemize}
		\item Planner
		\item GanttProject
		\item Trello + \textit{plugins}. En el siguiente link hay un tutorial oficial: \\
		      \url{https://blog.trello.com/es/diagrama-de-gantt-de-un-proyecto}
		\item Creately, herramienta online colaborativa.
		      \\\url{https://creately.com/diagram/example/ieb3p3ml/LaTeX}
		\item Se puede hacer en latex con el paquete \textit{pgfgantt}\\
		      \url{http://ctan.dcc.uchile.cl/graphics/pgf/contrib/pgfgantt/pgfgantt.pdf}
	\end{itemize}

	Pegar acá una captura de pantalla del diagrama de Gantt, cuidando que la letra
	sea suficientemente grande como para ser legible. Si el diagrama queda
	demasiado ancho, se puede pegar primero la ``tabla'' del Gantt y luego pegar la
	parte del diagrama de barras del diagrama de Gantt.

	Configurar el software para que en la parte de la tabla muestre los códigos del
	EDT (WBS).\\ Configurar el software para que al lado de cada barra muestre el
	nombre de cada tarea.\\ Revisar que la fecha de finalización coincida con lo
	indicado en el Acta Constitutiva.

	En la figura \ref{fig:gantt}, se muestra un ejemplo de diagrama de gantt
	realizado con el paquete de \textit{pgfgantt}. En la plantilla pueden ver el
	código que lo genera y usarlo de base para construir el propio.

	Las fechas pueden ser calculadas utilizando alguna de las herramientas antes
	citadas. Sin embargo, el siguiente ejemplo fue elaborado utilizando
	\href{https://docs.google.com/spreadsheets/d/1fBz8NhSpc4tkkhz3KjJCbh1nR_ltDkfEcZi4tZXduqs}{esta
		hoja de cálculo}.

	Es importante destacar que el ancho del diagrama estará dado por la longitud
	del texto utilizado para las tareas (Ejemplo: tarea 1, tarea 2, etcétera) y el
	valor \textit{x unit}. Para mejorar la apariencia del diagrama, es necesario
	ajustar este valor y, quizás, acortar los nombres de las tareas.

	\begin{landscape}
		\begin{figure}[htpb]
			\begin{center}
				\begin{ganttchart}[
						time slot unit=day,
						time slot format=isodate,
						x unit=0.3cm, % Ajusta el ancho de cada día
						y unit title=0.9cm, % Aumenta la altura de los títulos
						y unit chart=0.8cm, % Aumenta la altura de las filas del gráfico
						milestone/.append style={xscale=4},
						vgrid,
						hgrid,
						compress calendar % Permite ajustar el ancho de los días
						% today=2024-08-20, % Desactiva la línea de TODAY quitando o comentando esta línea
					]{2024-08-20}{2024-09-10}
					\gantttitlecalendar*{2024-08-20}{2024-09-10}{year} \\
					\gantttitlecalendar*{2024-08-20}{2024-09-10}{month} \\
					\gantttitlecalendar*{2024-08-20}{2024-09-10}{day} \\ % Agrega los días
	
					\ganttgroup{Planificación del proyecto}{2024-08-20}{2024-09-10} \\
					\ganttbar{Elaboración del documento de planificación}{2024-08-20}{2024-09-05} \\
					\ganttbar{Diseño de la arquitectura del proyecto}{2024-09-06}{2024-09-10} \\
	
				\end{ganttchart}
			\end{center}
			\caption{Diagrama de Gantt del Proyecto}
			\label{fig:gantt}
		\end{figure}
	\end{landscape}

	\begin{landscape}
		\begin{figure}[htpb]
			\centering
			\includegraphics[height=.85\textheight]{./Figuras/Gantt-2.png}
			\caption{Ejemplo de diagrama de Gantt (apaisado).} %Modificar este título acorde.
			\label{fig:diagGantt}
		\end{figure}

	\end{landscape}

\end{consigna}

En la figura \ref{fig:gantt}, se muestra un ejemplo de diagrama de gantt
realizado con el paquete de \textit{pgfgantt}. En la plantilla pueden ver el
código que lo genera y usarlo de base para construir el propio.

Las fechas pueden ser calculadas utilizando alguna de las herramientas antes
citadas. Sin embargo, el siguiente ejemplo fue elaborado utilizando
\href{https://docs.google.com/spreadsheets/d/1fBz8NhSpc4tkkhz3KjJCbh1nR_ltDkfEcZi4tZXduqs}{esta
	hoja de cálculo}.

Es importante destacar que el ancho del diagrama estará dado por la longitud
del texto utilizado para las tareas (Ejemplo: tarea 1, tarea 2, etcétera) y el
valor \textit{x unit}. Para mejorar la apariencia del diagrama, es necesario
ajustar este valor y, quizás, acortar los nombres de las tareas.

% Diagrama de Gantt Planificación del proyecto
\begin{landscape}
	\begin{figure}[htpb]
		\begin{center}
			\begin{ganttchart}[
					time slot unit=day,
					time slot format=isodate,
					x unit=0.5cm, % Aumenta el ancho de las barras
					y unit title=0.7cm, % Aumenta la altura de los títulos
					y unit chart=0.7cm, % Aumenta la altura de las filas del gráfico
					milestone/.append style={xscale=4},
					vgrid,
					hgrid,
				]{2024-08-20}{2024-09-10}
				\gantttitlecalendar*{2024-08-20}{2024-09-10}{year} \\
				\gantttitlecalendar*{2024-08-20}{2024-09-10}{month} \\
				\gantttitlecalendar*{2024-08-20}{2024-09-10}{day} \\

				\ganttgroup{Planificación del proyecto}{2024-08-20}{2024-09-10} \\
				\ganttbar{Elaboración del documento de planificación}{2024-08-20}{2024-09-05} \\
				\ganttbar{Diseño de la arquitectura del proyecto}{2024-09-06}{2024-09-10} \\
			\end{ganttchart}
		\end{center}
		\caption{Diagrama de Gantt del Proyecto}
		\label{fig:gantt1}
	\end{figure}
\end{landscape}

% Diagrama de Gantt Investigación preliminar
\begin{landscape}
	\begin{figure}[htpb]
		\begin{center}
			\begin{ganttchart}[
					time slot unit=day,
					time slot format=isodate,
					x unit=0.5cm, % Aumenta el ancho de las barras
					y unit title=0.7cm, % Aumenta la altura de los títulos
					y unit chart=0.7cm, % Aumenta la altura de las filas del gráfico
					milestone/.append style={xscale=4},
					vgrid,
					hgrid,
				]{2024-09-11}{2024-10-13}
				\gantttitlecalendar*{2024-09-11}{2024-10-13}{year} \\
				\gantttitlecalendar*{2024-09-11}{2024-10-13}{month} \\
				\gantttitlecalendar*{2024-09-11}{2024-10-13}{day} \\

				\ganttgroup{Investigación preliminar}{2024-09-11}{2024-10-13} \\
				\ganttbar{Investigación de protocolos}{2024-09-11}{2024-09-16} \\
				\ganttbar{Investigación de  TLS}{2024-09-17}{2024-09-18} \\
				\ganttbar{Investigación de microcontroladores}{2024-09-19}{2024-09-20} \\
				\ganttbar{Investigación de sensores y actuadores}{2024-09-21}{2024-09-24} \\
				\ganttbar{Investigación de la base de datos}{2024-09-25}{2024-09-28} \\
				\ganttbar{Investigación del framework backend}{2024-09-29}{2024-10-02} \\
				\ganttbar{Investigación del framework frontend}{2024-10-03}{2024-10-06} \\
				\ganttbar{Investigación del servidor IoT}{2024-10-07}{2024-10-10} \\
				\ganttbar{Instalación del entorno de desarrollo}{2024-10-11}{2024-10-13} \\
			\end{ganttchart}
		\end{center}
		\caption{Diagrama de Gantt del Proyecto}
		\label{fig:gantt1}
	\end{figure}
\end{landscape}




\begin{landscape}
	\begin{figure}[htpb]
		\begin{center}
			\begin{ganttchart}[
					time slot unit=day,
					time slot format=isodate,
					x unit=0.038cm,
					y unit title=0.7cm,
					y unit chart=0.6cm,
					milestone/.append style={xscale=4},
					vgrid,
					hgrid,
					today=2024-08-20]{2024-10-21}{2024-11-10}
				\gantttitlecalendar*{2024-10-21}{2024-11-10}{year} \\
				\gantttitlecalendar*{2024-10-21}{2024-11-10}{month} \\

				\ganttgroup{Prototipado del proyecto}{2024-10-21}{2024-11-10} \\
				\ganttbar{Configuración y conexionado de sensores y actuadores}{2024-10-21}{2024-10-31} \\
				\ganttbar{Integración de componentes y microcontroladores}{2024-11-01}{2024-11-10} \\
			\end{ganttchart}
		\end{center}
		\caption{Diagrama de Gantt del Proyecto}
		\label{fig:gantt3}
	\end{figure}
\end{landscape}

\begin{landscape}
	\begin{figure}[htpb]
		\begin{center}
			\begin{ganttchart}[
					time slot unit=day,
					time slot format=isodate,
					x unit=0.038cm,
					y unit title=0.7cm,
					y unit chart=0.6cm,
					milestone/.append style={xscale=4},
					vgrid,
					hgrid,
					today=2024-08-20
				]{2024-08-20}{2025-06-30}
				\gantttitlecalendar*{2024-11-11}{2025-02-20}{year} \\
				\gantttitlecalendar*{2024-11-11}{2025-02-20}{month} \\

				\ganttgroup{Desarrollo del firmware}{2024-11-11}{2025-01-10} \\
				\ganttbar{Configuración inicial de los microcontroladores}{2024-11-11}{2024-11-12} \\
				\ganttbar{Implementación y configuración de certificados TLS}{2024-11-13}{2024-11-22} \\
				\ganttbar{Firmware nodo sensor de temperatura, humedad, presión y luz}{2024-11-23}{2024-12-07} \\
				\ganttbar{Firmware nodo sensor de dióxido de carbono}{2024-12-08}{2024-12-22} \\
				\ganttbar{Firmware nodo sensor de pH, CE y TDS}{2024-12-23}{2025-01-06} \\
				\ganttbar{Firmware nodo sensor de solución nutritiva}{2025-01-07}{2025-01-21} \\
				\ganttbar{Firmware nodo sensor de agua, nutrientes y energía eléctrica}{2025-01-22}{2025-02-05} \\
				\ganttbar{Firmware nodo actuador}{2025-02-06}{2025-02-20} \\

			\end{ganttchart}
		\end{center}
		\caption{Diagrama de Gantt del Proyecto}
		\label{fig:gantt}
	\end{figure}
\end{landscape}

\begin{landscape}
	\begin{figure}[htpb]
		\begin{center}
			\begin{ganttchart}[
					time slot unit=day,
					time slot format=isodate,
					x unit=0.038cm,
					y unit title=0.7cm,
					y unit chart=0.6cm,
					milestone/.append style={xscale=4},
					vgrid,
					hgrid,
					today=2024-08-20
				]{2024-08-20}{2025-06-30}
				\gantttitlecalendar*{2025-02-21}{2025-04-10}{year} \\
				\gantttitlecalendar*{2025-02-21}{2025-05-31}{month} \\

				\ganttgroup{Desarrollo del backend}{2025-02-21}{2025-04-10} \\
				\ganttbar{Configuración inicial del entorno}{2025-02-21}{2025-02-28} \\
				\ganttbar{Implementación del broker MQTT}{2025-03-01}{2025-03-10} \\
				\ganttbar{Configuración de certificados TLS y JWT}{2025-03-11}{2025-03-20} \\
				\ganttbar{Configuración y prueba de la base de datos}{2025-03-21}{2025-03-30} \\
				\ganttbar{Implementación de WebSockets y endpoints}{2025-03-31}{2025-04-10} \\

				\ganttgroup{Desarrollo del frontend}{2025-04-11}{2025-05-31} \\
				\ganttbar{Configuración del entorno y TLS}{2025-04-11}{2025-04-20} \\
				\ganttbar{Implementación de JWT y WebSockets}{2025-04-21}{2025-05-01} \\
				\ganttbar{Desarrollo de interfaces de usuario}{2025-05-02}{2025-05-31} \\
			\end{ganttchart}
		\end{center}
		\caption{Diagrama de Gantt del Proyecto}
		\label{fig:gantt}
	\end{figure}
\end{landscape}

\begin{landscape}
	\begin{figure}[htpb]
		\begin{center}
			\begin{ganttchart}[
					time slot unit=day,
					time slot format=isodate,
					x unit=0.038cm,
					y unit title=0.7cm,
					y unit chart=0.6cm,
					milestone/.append style={xscale=4},
					vgrid,
					hgrid,
					today=2024-08-20
				]{2024-08-20}{2025-06-30}
				\gantttitlecalendar*{2024-06-01}{2025-06-30}{year} \\
				\gantttitlecalendar*{2024-06-01}{2025-06-30}{month} \\

				\ganttgroup{Pruebas y validación}{2025-06-01}{2025-06-15} \\
				\ganttbar{Pruebas de microcontroladores y firmware}{2025-06-01}{2025-06-05} \\
				\ganttbar{Pruebas de backend y frontend}{2025-06-06}{2025-06-10} \\
				\ganttbar{Pruebas de seguridad y rendimiento}{2025-06-11}{2025-06-15} \\

				\ganttgroup{Documentación del proyecto}{2025-06-16}{2025-06-30} \\
				\ganttbar{Informe de avance y revisión}{2025-06-16}{2025-06-20} \\
				\ganttbar{Video demostración y presentación final}{2025-06-21}{2025-06-30} \\
			\end{ganttchart}
		\end{center}
		\caption{Diagrama de Gantt del Proyecto}
		\label{fig:gantt}
	\end{figure}
\end{landscape}







\begin{table}[ht]
	\begin{tabularx}{\linewidth}{|p{1cm}|p{10cm}|p{1cm}|p{1.8cm}|p{1.8cm}|}
		\hline
		\rowcolor[HTML]{C0C0C0}
		WBS  & Tarea                                                                                                    & Horas & Inicio     & Fin        \\ \hline
		1    & Planificación del proyecto                                                                               & 50    & 20-08-2024 & 07-09-2024 \\ \hline
		1.1  & Elaboración del documento de planificación del proyecto                                                  & 35    & 20-08-2024 & 31-08-2024 \\ \hline
		1.2  & Diseño de la arquitectura del proyecto                                                                   & 15    & 02-09-2024 & 07-09-2024 \\ \hline
		2    & Investigación preliminar                                                                                 & 80    & 09-09-2024 & 19-10-2024 \\ \hline
		2.1  & Investigación de los protocolos MQTT, HTTP y WebSockets                                                  & 12    & 09-09-2024 & 13-09-2024 \\ \hline
		2.2  & Investigación de certificado TLS                                                                         & 8     & 14-09-2024 & 17-09-2024 \\ \hline
		2.3  & Investigación y elección de los microcontroladores a utilizar                                            & 5     & 18-09-2024 & 20-09-2024 \\ \hline
		2.4  & Investigación y elección de los sensores y actuadores a utilizar                                         & 10    & 21-09-2024 & 26-09-2024 \\ \hline
		2.5  & Investigación y elección de la base de datos a utilizar                                                  & 10    & 27-09-2024 & 02-10-2024 \\ \hline
		2.6  & Investigación y elección del framework de backend a utilizar                                             & 10    & 03-10-2024 & 08-10-2024 \\ \hline
		2.7  & Investigación y elección del framework de frontend a utilizar                                            & 10    & 09-10-2024 & 14-10-2024 \\ \hline
		2.8  & Investigación y elección de la plataforma del servidor IoT a utilizar                                    & 10    & 15-10-2024 & 20-10-2024 \\ \hline
		2.9  & Instalación y puesta a punto del entorno de desarrollo                                                   & 5     & 21-10-2024 & 22-10-2024 \\ \hline
		3    & Prototipado del proyecto                                                                                 & 30    & 23-10-2024 & 04-11-2024 \\ \hline
		3.1  & Configuración y conexionado de sensores y actuadores                                                     & 15    & 23-10-2024 & 29-10-2024 \\ \hline
		3.2  & Integración de componentes y microcontroladores                                                          & 15    & 30-10-2024 & 04-11-2024 \\ \hline
		4    & Desarrollo del firmware de los microcontroladores                                                        & 127   & 05-11-2024 & 30-12-2024 \\ \hline
		4.1  & Configuración inicial de los microcontroladores                                                          & 2     & 05-11-2024 & 05-11-2024 \\ \hline
		4.2  & Implementación y configuración de los certificados TLS                                                   & 10    & 06-11-2024 & 09-11-2024 \\ \hline
		4.3  & Implementación del firmware del nodo sensor de temp. amb., hum. rel., pres. atmo. y nivel de luminosidad & 20    & 11-11-2024 & 16-11-2024 \\ \hline
		4.4  & Implementación del firmware del nodo sensor de nivel de dióxido de carbono                               & 15    & 18-11-2024 & 23-11-2024 \\ \hline
		4.5  & Implementación del firmware del nodo sensor de pH, CE y TDS                                              & 20    & 25-11-2024 & 30-11-2024 \\ \hline
		4.6  & Implementación del firmware del nodo sensor de nivel y temperatura de la solución nutritiva              & 20    & 02-12-2024 & 07-12-2024 \\ \hline
		4.7  & Implementación del firmware del nodo sensor de agua, nutrientes y energía eléctrica                      & 20    & 09-12-2024 & 14-12-2024 \\ \hline
		4.8  & Implementación del firmware del nodo actuador                                                            & 20    & 16-12-2024 & 21-12-2024 \\ \hline
		5    & Desarrollo del backend                                                                                   & 80    & 02-02-2025 & 15-03-2025 \\ \hline
		5.1  & Configuración inicial del entorno                                                                        & 10    & 02-02-2025 & 05-02-2025 \\ \hline
		5.2  & Implementación y configuración del broker MQTT                                                           & 10    & 06-02-2025 & 09-02-2025 \\ \hline
		5.3  & Implementación y configuración de los certificados TLS                                                   & 10    & 10-02-2025 & 13-02-2025 \\ \hline
		5.4  & Implementación y configuración de JWT                                                                    & 10    & 14-02-2025 & 18-02-2025 \\ \hline
		5.5  & Implementación y configuración de la base de datos                                                       & 10    & 19-02-2025 & 22-02-2025 \\ \hline
		5.6  & Implementación y configuración de WebSockets                                                             & 10    & 23-02-2025 & 26-02-2025 \\ \hline
		5.7  & Implementación de los endpoints de usuarios                                                              & 4     & 27-02-2025 & 28-02-2025 \\ \hline
		5.8  & Pruebas de los endpoints de usuarios                                                                     & 1     & 01-03-2025 & 01-03-2025 \\ \hline
		5.9  & Implementación de los endpoints de ambientes                                                             & 2     & 03-03-2025 & 04-03-2025 \\ \hline
		5.10 & Pruebas de los endpoints de ambientes                                                                    & 1     & 05-03-2025 & 05-03-2025 \\ \hline
		5.11 & Implementación de los endpoints de sensores                                                              & 4     & 06-03-2025 & 07-03-2025 \\ \hline
		5.12 & Pruebas de los endpoints de sensores                                                                     & 1     & 08-03-2025 & 08-03-2025 \\ \hline
		5.13 & Implementación de los endpoints de actuadores                                                            & 4     & 10-03-2025 & 11-03-2025 \\ \hline
		5.14 & Pruebas de los endpoints de actuadores                                                                   & 1     & 12-03-2025 & 12-03-2025 \\ \hline
		5.15 & Implementación del endpoint de mediciones de sensores                                                    & 2     & 13-03-2025 & 13-03-2025 \\ \hline
	\end{tabularx}
	\caption{Lista de tareas del proyecto con fechas.}
	\label{tab:tabGantt}
\end{table}

\begin{table}[ht]
	\begin{tabularx}{\linewidth}{|p{1cm}|p{10cm}|p{1cm}|p{1.8cm}|p{1.8cm}|}
		\hline
		\rowcolor[HTML]{C0C0C0}
		WBS   & Tarea                                                                 & Horas & Inicio     & Fin        \\ \hline
		5.16  & Pruebas del endpoint de mediciones de sensores                        & 1     & 14-03-2025 & 14-03-2025 \\ \hline
		5.17  & Implementación del endpoint de estados de actuadores                  & 2     & 15-03-2025 & 15-03-2025 \\ \hline
		5.18  & Pruebas del endpoint de estados de actuadores                         & 1     & 16-03-2025 & 16-03-2025 \\ \hline
		5.19  & Implementación del endpoint de parámetros enviados a los sensores     & 2     & 17-03-2025 & 17-03-2025 \\ \hline
		5.20  & Pruebas del endpoint de parámetros enviados a los sensores            & 1     & 18-03-2025 & 18-03-2025 \\ \hline
		5.21  & Implementación del endpoint de parámetros enviados a los actuadores   & 2     & 20-03-2025 & 20-03-2025 \\ \hline
		5.22  & Pruebas del endpoint de parámetros enviados a los actuadores          & 1     & 21-03-2025 & 21-03-2025 \\ \hline
		6     & Desarrollo del frontend                                               & 93    & 24-03-2025 & 08-06-2025 \\ \hline
		6.1   & Configuración inicial del entorno                                     & 10    & 24-03-2025 & 27-03-2025 \\ \hline
		6.2   & Implementación y configuración de los certificados TLS                & 10    & 28-03-2025 & 01-04-2025 \\ \hline
		6.3   & Implementación y configuración de JWT                                 & 10    & 02-04-2025 & 05-04-2025 \\ \hline
		6.4   & Implementación y configuración de WebSockets                          & 5     & 07-04-2025 & 09-04-2025 \\ \hline
		6.5   & Implementación de la interfaz de login de usuario                     & 4     & 10-04-2025 & 11-04-2025 \\ \hline
		6.6   & Pruebas de la interfaz de login de usuario                            & 1     & 14-04-2025 & 14-04-2025 \\ \hline
		6.7   & Implementación de las interfaces de manejo de usuarios                & 4     & 15-04-2025 & 16-04-2025 \\ \hline
		6.8   & Pruebas de las interfaces de manejo de usuarios                       & 1     & 17-04-2025 & 17-04-2025 \\ \hline
		6.9   & Implementación de las interfaces de manejo de ambientes               & 3     & 18-04-2025 & 21-04-2025 \\ \hline
		6.10  & Pruebas de las interfaces de manejo de ambientes                      & 1     & 22-04-2025 & 22-04-2025 \\ \hline
		6.11  & Implementación de las interfaces de manejo de sensores                & 5     & 23-04-2025 & 27-04-2025 \\ \hline
		6.12  & Pruebas de las interfaces de manejo de sensores                       & 1     & 28-04-2025 & 28-04-2025 \\ \hline
		6.13  & Implementación de las interfaces de manejo de actuadores              & 5     & 29-04-2025 & 04-05-2025 \\ \hline
		6.14  & Pruebas de las interfaces de manejo de actuadores                     & 1     & 05-05-2025 & 05-05-2025 \\ \hline
		6.15  & Implementación de la interfaz de mediciones de sensores               & 8     & 06-05-2025 & 13-05-2025 \\ \hline
		6.16  & Pruebas de la interfaz de mediciones de sensores                      & 1     & 14-05-2025 & 14-05-2025 \\ \hline
		6.17  & Implementación de la interfaz de estados de actuadores                & 8     & 15-05-2025 & 22-05-2025 \\ \hline
		6.18  & Pruebas de la interfaz de estados de actuadores                       & 1     & 23-05-2025 & 23-05-2025 \\ \hline
		6.19  & Implementación de la interfaz de parámetros enviados a los sensores   & 3     & 26-05-2025 & 28-05-2025 \\ \hline
		6.20  & Pruebas de la interfaz de parámetros enviados a los sensores          & 1     & 29-05-2025 & 29-05-2025 \\ \hline
		6.21  & Implementación de la interfaz de parámetros enviados a los actuadores & 3     & 30-05-2025 & 03-06-2025 \\ \hline
		6.22  & Pruebas de la interfaz de parámetros enviados a los actuadores        & 1     & 04-06-2025 & 04-06-2025 \\ \hline
		6.23  & Implementación de la interfaz principal del sistema                   & 5     & 05-06-2025 & 10-06-2025 \\ \hline
		6.24  & Pruebas de la interfaz de la interfaz principal del sistema           & 1     & 11-06-2025 & 11-06-2025 \\ \hline
		7     & Pruebas y validación                                                  & 47    & 12-06-2025 & 26-06-2025 \\ \hline
		7.1   & Pruebas de los microcontroladores                                     &       &            &            \\ \hline
		7.1.1 & Pruebas de los firmwares de los microcontroladores                    & 5     & 12-06-2025 & 14-06-2025 \\ \hline
		7.1.2 & Pruebas de conexión y comunicación de los microcontroladores          & 5     & 15-06-2025 & 18-06-2025 \\ \hline
		7.1.3 & Pruebas de seguridad                                                  & 5     & 19-06-2025 & 21-06-2025 \\ \hline
		7.1.4 & Pruebas de sensores y actuadores                                      & 5     & 22-06-2025 & 25-06-2025 \\ \hline
		7.2   & Pruebas del backend                                                   &       &            &            \\ \hline
		7.2.1 & Pruebas de MQTT y TLS                                                 & 3     & 26-06-2025 & 28-06-2025 \\ \hline
		7.2.2 & Pruebas de validación de almacenamiento de los datos                  & 3     & 29-06-2025 & 01-07-2025 \\ \hline
	\end{tabularx}
	\caption{Lista de tareas del proyecto con fechas.}
	\label{tab:tabGantt}
\end{table}

\begin{table}[ht]
	\begin{tabularx}{\linewidth}{|p{1cm}|p{10cm}|p{1cm}|p{1.8cm}|p{1.8cm}|}
		\hline
		\rowcolor[HTML]{C0C0C0}
		WBS   & Tarea                                                     & Horas & Inicio     & Fin        \\ \hline
		7.2.3 & Pruebas de manejo de los datos en tiempo real y endpoints & 3     & 02-07-2025 & 04-07-2025 \\ \hline
		7.2.4 & Pruebas de seguridad                                      & 5     & 05-07-2025 & 09-07-2025 \\ \hline
		7.3   & Pruebas del frontend                                      &       &            &            \\ \hline
		7.3.1 & Pruebas de interfaz y usabilidad                          & 3     & 10-07-2025 & 12-07-2025 \\ \hline
		7.3.2 & Pruebas de visualización de los reportes y gráficos       & 3     & 13-07-2025 & 15-07-2025 \\ \hline
		7.3.3 & Pruebas de visualización de los datos en tiempo real      & 3     & 16-07-2025 & 18-07-2025 \\ \hline
		7.3.4 & Pruebas de seguridad                                      & 5     & 19-07-2025 & 23-07-2025 \\ \hline
		8     & Documentación del proyecto                                & 100   & 24-07-2025 & 16-09-2025 \\ \hline
		8.1   & Informe de avance                                         & 15    & 24-07-2025 & 30-07-2025 \\ \hline
		8.2   & Video con demostración del sistema                        & 2     & 31-07-2025 & 01-08-2025 \\ \hline
		8.3   & Memoria del proyecto                                      & 50    & 02-08-2025 & 01-09-2025 \\ \hline
		8.4   & Revisión de memoria del proyecto                          & 10    & 02-09-2025 & 08-09-2025 \\ \hline
		8.5   & Presentación final                                        & 20    & 09-09-2025 & 16-09-2025 \\ \hline
	\end{tabularx}
	\caption{Lista de tareas del proyecto con fechas.}
	\label{tab:tabGantt}
\end{table}

\begin{itemize}
	\item Cliente: El \clientename\hspace{1px} es especialista en cultivos y gestión
	      forestal, con amplia experiencia en cultivos hidropónicos. Va a colaborar con
	      la definición de los requerimientos y el seguimiento del proyecto.
	\item Orientadores:
	      \begin{itemize}
		      \item El Director \supname\hspace{1px} es experto en la temática y guiará con la
		            implementación de los protocolos y herramientas del proyecto.
		      \item La Codirectora \cosupname , experta en Sistemas de Información, ayudará con el
		            seguimiento metodológico, para garantizar una gestión rigurosa y efectiva del
		            desarrollo del trabajo.
	      \end{itemize}
\end{itemize}
