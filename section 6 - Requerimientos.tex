\begin{enumerate}
	\item Requerimientos de los nodos sensores:
	      \begin{enumerate}
		      \item Cada nodo deberá contar con un microcontrolador ESP32-C3.
		      \item Cada nodo deberá conectarse a una red Wi-Fi.
		      \item Cada nodo deberá tener un identificador único dentro del sistema.
		      \item Cada nodo deberá permitir configurar el tiempo de manera remota para enviar los datos recolectados.
		      \item Cada nodo, según corresponda, deberá enviar al servidor IoT:
		            \begin{enumerate}
			            \item la temperatura del ambiente, humedad relativa, presión atmosférica y nivel de luminosidad,
			            \item el nivel de dióxido de carbono,
			            \item el nivel de pH, CE y TDS,
			            \item el nivel y temperatura de la solución nutritiva,
			            \item y consumo de agua, nutrientes y energía eléctrica.
		            \end{enumerate}
	      \end{enumerate}

	\item Requerimientos del nodo actuador:
	      \begin{enumerate}
		      \item Deberá contar con un microcontrolador ESP32-C3.
		      \item Cada nodo deberá conectarse a una red Wi-Fi.
		      \item Deberá tener un identificador único dentro del sistema.
		      \item Deberá contar con cuatro canales.
		      \item Deberá enviar al servidor IoT el estado de cada canal.
		      \item Deberá permitir activar cada canal de manera remota.
	      \end{enumerate}

	\item Requerimientos asociados al Broker MQTT:
	      \begin{enumerate}
		      \item Deberá contar con un broker MQTT que soporte conexiones seguras mediante TLS.
		      \item Deberá gestionar las suscripciones y publicaciones de los datos enviados desde
		            los nodos y permitir la comunicación bidireccional para el envío de comandos.
		      \item Deberá implementar QoS (\textit{Quality of Service}) para garantizar la entrega de los mensajes.
	      \end{enumerate}

	\item Requerimientos asociados al Frontend:
	      \begin{enumerate}
		      \item La interfaz deberá ser intuitiva y accesible desde dispositivos móviles y de escritorio.
		      \item Deberá permitir el acceso al sistema a usuarios debidamente autenticados.
		      \item Deberá permitir realizar el CRUD (\textit{Create, Read, Update and Delete}) de ambientes.
		      \item Deberá permitir realizar el CRUD de sensores asociados a un determinado ambiente.
		      \item Deberá permitir realizar el CRUD de actuadores asociados a un determinado ambiente.
		      \item Deberá permitir visualizar en tiempo real los datos recibidos de los nodos sensores.
		      \item Deberá permitir enviar comandos y configuraciones a los nodos sensores y actuadores.
		      \item Deberá permitir visualizar los datos históricos de los sensores y actuadores.
	      \end{enumerate}

	\item Requerimientos asociados al Backend y la API:
	      \begin{enumerate}
		      \item Deberá permitir conexiones seguras mediante TLS.
		      \item Deberá soportar los métodos HTTP para realizar operaciones CRUD y visualizar los reportes, WebSockets
		            para la visualización en tiempo real de los datos y MQTT para la interacción con los nodos.
			  \item Deberá poder persistir la información de los usuarios.
			  \item Deberá poder persistir la información de los ambientes.
			  \item Deberá poder persistir la información de los sensores.
			  \item Deberá poder persistir la información de los actuadores.
			  \item Deberá poder persistir la información histórica de las mediciones de los sensores.
			  \item Deberá poder persistir la información histórica de los estados de los actuadores.
			  \item Deberá poder persistir la información histórica de los parámetros enviados a los sensores.
			  \item Deberá poder persistir la información histórica de los parámetros enviados a los actuadores.
		      \item Deberá permitir la autenticación y autorización de usuarios autorizados para acceder a los recursos.
		      \item Deberá incluir los endpoints para permitir realizar el CRUD de usuarios para el acceso seguro al sistema.
		      \item Deberá incluir los endpoints para permitir realizar el CRUD de ambientes.
		      \item Deberá incluir los endpoints para permitir realizar el CRUD de sensores asociados a un determinado ambiente, 
			  con la configuración de sus parámetros específicos.
		      \item Deberá incluir los endpoints para permitir realizar el CRUD de actuadores asociados a un determinado ambiente, 
			  con la configuración de sus parámetros específicos por canales.
		      \item Deberá incluir el endpoint para registrar las mediciones de los sensores.
		      \item Deberá incluir el endpoint para registrar los estados de los actuadores.
		      \item Deberá incluir el endpoint para obtener el histórico de las mediciones de los sensores.
		      \item Deberá incluir el endpoint para obtener el histórico de los estados de los actuadores.
		      \item Deberá incluir un endpoint para el envío de parámetros a los sensores.
		      \item Deberá incluir un endpoint para el envío de parámetros a los actuadores.
	      \end{enumerate}

	\item Requerimientos de documentación:
	      \begin{enumerate}
		      \item Se entregará el código del sistema, que incluye todos los componentes desarrollados (sensores, actuadores, Broker MQTT, Frontend, Backend y API).
		      \item Se entregarán las guías de instalación, configuración y operación.
		      \item Se desarrollará un informe de avance al finalizar el Taller de Trabajo Final A.
		      \item Se desarrollará la memoria del proyecto al finalizar el Taller de Trabajo Final B.
	      \end{enumerate}
\end{enumerate}
