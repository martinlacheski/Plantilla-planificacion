Para definir las historias de usuario y estimar su dificultad, complejidad y riesgo, es necesario primero descomponer los requerimientos en historias de
usuario específicas. A cada historia se le asignan puntos de acuerdo con la serie de Fibonacci (1, 2, 3, 5, 8, 13, 21, 34, etc.), que representan los niveles
de dificultad, complejidad y riesgo.

El puntaje total de cada historia de usuario se calcula con la suma de los valores asignados a estos tres criterios, y luego se asigna el número de la
secuencia de Fibonacci más cercano que sea mayor o igual a la suma obtenida. 

Para asignar ponderación a los (\textit{history points}) se han asignado los
siguientes pesos:

\underline{Dificultad del trabajo a realizar}
\begin{itemize}
	\item \textbf{Baja:} Peso \textrightarrow{ 1}.
	\item \textbf{Media:} Peso \textrightarrow{ 3}.
	\item \textbf{Alta:} Peso \textrightarrow{ 5}.
\end{itemize}

\underline{Complejidad del trabajo a realizar}
\begin{itemize}
	\item \textbf{Baja:} Peso \textrightarrow{ 1}.
	\item \textbf{Media:} Peso \textrightarrow{ 5}.
	\item \textbf{Alta:} Peso \textrightarrow{ 13}.
\end{itemize}

\underline{Riesgo del trabajo a realizar}
\begin{itemize}
	\item \textbf{Baja:} Peso \textrightarrow{ 1}.
	\item \textbf{Media:} Peso \textrightarrow{ 3}.
	\item \textbf{Alta:} Peso \textrightarrow{ 5}.
\end{itemize}

\pagebreak

\textbf{Historia de Usuario 1 \textrightarrow{ Desarrollador}}
\begin{itemize}
	\item ``Como desarrollador, quiero integrar los sensores y actuadores con el sistema para que se comuniquen correctamente a través de la
	      red y asegurar que los datos se transmitan de manera eficiente y segura."

	      \begin{itemize}
		      \item \textbf{Dificultad:} Alta (5) \textrightarrow{ Porque implica múltiples integraciones y la configuración de varios dispositivos para que
			            funcionen en conjunto de manera sincronizada}.
		      \item \textbf{Complejidad:} Alta (13) \textrightarrow{ Porque la integración de todas las tecnologías podría presentar desafíos técnicos}.
		      \item \textbf{Riesgo:} Alta (5) \textrightarrow{ Porque existe incertidumbre sobre posibles problemas de comunicación entre dispositivos
			            y ajustes de la configuración}.
	      \end{itemize}

	      \textit{Story points}= 34.

	      (5 + 13 + 5 = 23 \textrightarrow{ 34}). Es el siguiente valor en la serie de Fibonacci.

\end{itemize}

\textbf{Historia de Usuario 2 \textrightarrow{ Cliente}}
\begin{itemize}
	\item ``Como cliente, quiero que el sistema permita la gestión completa de ambientes, sensores y actuadores a través de una interfaz intuitiva,
	      con capacidades para visualizar en tiempo real, ajustar las configuraciones de manera remota y visualizar
	      reportes históricos."

	      \begin{itemize}
		      \item \textbf{Dificultad:} Media (3) \textrightarrow{ Porque implica el desarrollo y personalización de funcionalidades estándar que son comunes
			            en sistemas de monitoreo y control}.
		      \item \textbf{Complejidad:} Alta (13) \textrightarrow{ Si bien existen soluciones de este tipo y se podría consultar a profesionales expertos en la
			            materia, integrar todas las funcionalidades en una interfaz coherente y eficiente podría ser complejo}.
		      \item \textbf{Riesgo:} Media (3) \textrightarrow{ Porque la integración de múltiples módulos y la gestión de datos en tiempo real presenta riesgos
			            moderados que podrían afectar la estabilidad del sistema}.
	      \end{itemize}

	      \textit{Story points}= 21.

	      (3 + 13 + 3 = 19 \textrightarrow{ 21}). Es el siguiente valor en la serie de Fibonacci.

\end{itemize}

\textbf{Historia de Usuario 3 \textrightarrow{ Usuario final}}
\begin{itemize}
	\item ``Como usuario final, quiero poder visualizar un panel de control que sea fácil de usar, responsivo desde dispositivos móviles y de escritorio,
	      que me permita gestionar y visualizar datos de sensores y actuadores en tiempo real, además de consultar reportes históricos."

	      \begin{itemize}
		      \item \textbf{Dificultad:} Media (3) \textrightarrow{ Aunque existen herramientas para el desarrollo, la integración de funcionalidades avanzadas
			            y la necesidad de una interfaz de usuario adaptativa para diferentes dispositivos pueden añadir dificultad}.
		      \item \textbf{Complejidad:} Alta (13) \textrightarrow{ El diseño y la implementación de una interfaz responsiva que soporte múltiples funcionalidades
			            (monitoreo en tiempo real, datos históricos, gestión remota) pueden requerir un desarrollo más complejo}.
		      \item \textbf{Riesgo:} Media (3) \textrightarrow{ Si bien existen las tecnologías para desarrollar el proyecto, la implementación exitosa de todas las
			            funcionalidades sin problemas técnicos  y la integración con otras partes del sistema pueden presentar riesgos}.
	      \end{itemize}

	      \textit{Story points}= 21.

	      (3 + 13 + 3 = 19 \textrightarrow{ 21}). Es el siguiente valor en la serie de Fibonacci.

\end{itemize}