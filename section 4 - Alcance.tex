El presente trabajo incluye:
\begin{itemize}
	\item Diseño e implementación de la conexión Wi-Fi en los nodos.
	\item Desarrollo del firmware de los nodos.
	      \begin{itemize}
		      \item Programación del firmware de los siguentes nodos sensores para la recopilación
		            y transmisión de datos.
		            \begin{itemize}
			            \item Nodo sensor de temperatura ambiente, humedad relativa, presión atmosférica y nivel de luminosidad en el invernadero.
			            \item Nodo sensor de dióxido de carbono ($CO_2$).
			            \item Nodo sensor de pH, CE y Total de Sólidos Disueltos (TDS).
			            \item Nodo sensor de nivel y temperatura de la solución nutritiva.
			            \item Nodo sensor de consumo de agua, nutrientes y energía eléctrica.
		            \end{itemize}
		      \item Programación del firmware del nodo actuador de cuatro salidas para gestionar el
		            control de dispositivos y reportar su estado.
	      \end{itemize}
	\item Implementación del broker MQTT en el servidor IoT.
	\item Implementación de cifrados de conexión mediante TLS (\textit{Transport Layer Security}).
	\item Configuración del nodo central para enviar los datos recolectados al servidor IoT mediante el protocolo MQTT.
	\item Diseño e implementación de una base de datos para almacenar los datos recolectados por los sensores y permitir su consulta y análisis.
	\item Diseño y desarrollo de una API (\textit{Application Programming Interface}) REST (\textit{Representational State Transfer}) que permita 
		  la comunicación con el sistema utilizando HTTP (\textit{Hypertext Transfer Protocol}), MQTT y WebSockets.
	\item Desarrollo de una aplicación Web del tipo SPA.
	\item Entrega del código del sistema, que incluye todos los componentes desarrollados (sensores, actuadores, servidor y aplicación web), guías de instalación,
	      configuración y operación.
\end{itemize}

El presente trabajo no incluye:
\begin{itemize}
	\item Armado de PCB, se van a utilizar gabinetes estancos, protoboard y cables para conectar los microcontroladores con los sensores y actuadores.
	\item Desarrollo de una aplicación móvil compatible con iOS y Android.
\end{itemize}