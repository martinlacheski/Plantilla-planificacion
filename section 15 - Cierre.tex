Al finalizar el proyecto, se realizarán las siguientes actividades que permita
una evaluación del proyecto:

\begin{itemize}
	\item Análisis de las pautas de trabajo:
	      \begin{itemize}
		      \item Responsable: \authorname 
		      \item Procedimiento: se realizará una revisión del Plan de Proyecto para:
		            \begin{itemize}
			            \item Verificar si se cumplieron los plazos establecidos.
			            \item Verificar si se cumplieron los objetivos del proyecto.
			            \item Verificar si pudieron cumplimentarse y validarse los requerimientos definidos.
		            \end{itemize}
	      \end{itemize}
	\item Identificación de técnicas y procedimientos útiles e inútiles:
	      \begin{itemize}
		      \item Responsable: \authorname
		      \item Procedimiento: se realizará un análisis de los problemas que surgieron y cómo
		            se solucionaron para:
		            \begin{itemize}
			            \item Examinar los problemas que surgieron durante el proyecto y cómo se resolvieron,
			                  con el fin de identificar las técnicas y procedimientos que resultaron
			                  efectivos o ineficaces.
			            \item Documentar los problemas encontrados, las soluciones implementadas y las
			                  técnicas empleadas durante el desarrollo del proyecto.
			            \item Redactar un informe que analice las técnicas y procedimientos utilizados,
			                  destacar las prácticas que resultaron útiles y señalar las áreas de mejora.
			                  Este informe servirá como guía y referencia para la optimización en futuros
			                  proyectos.
		            \end{itemize}
	      \end{itemize}
	\item Acto de cierre:
	      \begin{itemize}
		      \item Responsable: \authorname
		      \item Procedimiento: se llevará a cabo un acto de cierre formal del proyecto, el cual
		            incluirá los siguientes pasos:
		            \begin{itemize}
			            \item Defensa pública del proyecto: se realizará una presentación pública del
			                  proyecto de manera virtual, donde se expondrán los objetivos, resultados y
			                  conclusiones del trabajo. Esta defensa permitirá mostrar los logros alcanzados
			                  y responder a preguntas formuladas por los participantes de la defensa.
			            \item Agradecimientos Formales:
			                  \begin{itemize}
				                  \item Directores del proyecto: se reconocerá y agradecerá el apoyo y la guía brindada
				                        durante el desarrollo del proyecto.
				                  \item Tribunal evaluador: se expresará el agradecimiento por la evaluación crítica y
				                        constructiva que contribuyó al perfeccionamiento del proyecto.
				                  \item Institución: se agradecerá a la institución que facilitó el entorno y los
				                        recursos necesarios para llevar a cabo el proyecto.
				                  \item Colaboradores: se reconocerá el esfuerzo y la contribución de todos los
				                        colaboradores que hicieron posible el éxito del proyecto, así como su
				                        compromiso y dedicación.
				                  \item Docentes y autoridades de la carrera: se dará un agradecimiento a los docentes
				                        y autoridades que proporcionaron apoyo académico y administrativo, y que
				                        jugaron un papel importante en la formación profesional para el desarrollo del
				                        proyecto.
			                  \end{itemize}
		            \end{itemize}
	      \end{itemize}
\end{itemize}