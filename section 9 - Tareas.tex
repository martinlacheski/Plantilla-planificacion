\begin{consigna}{red}
	El WBS debe tener relación directa o indirecta con los requerimientos.  Son todas las actividades que se harán en el proyecto para dar cumplimiento a los requerimientos. Se recomienda mostrar el WBS mediante una lista indexada:

	\begin{enumerate}
		\item Grupo de tareas 1 (suma h)
		      \begin{enumerate}
			      \item Tarea 1 (tantas h)
			      \item Tarea 2 (tantas h)
			      \item Tarea 3 (tantas h)
		      \end{enumerate}
		\item Grupo de tareas 2 (suma h)
		      \begin{enumerate}
			      \item Tarea 1 (tantas h)
			      \item Tarea 2 (tantas h)
			      \item Tarea 3 (tantas h)
		      \end{enumerate}
		\item Grupo de tareas 3 (suma h)
		      \begin{enumerate}
			      \item Tarea 1 (tantas h)
			      \item Tarea 2 (tantas h)
			      \item Tarea 3 (tantas h)
			      \item Tarea 4 (tantas h)
			      \item Tarea 5 (tantas h)
		      \end{enumerate}
	\end{enumerate}

	Cantidad total de horas: tantas.

	\textbf{¡Importante!:} la unidad de horas es h y va separada por espacio del número. Es incorrecto escribir ``23hs".

	\textbf{Se recomienda que no haya ninguna tarea que lleve más de 40 h.} De ser así se recomienda dividirla en tareas de menor duración.

\end{consigna}